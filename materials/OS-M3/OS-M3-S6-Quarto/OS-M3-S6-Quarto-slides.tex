% Options for packages loaded elsewhere
% Options for packages loaded elsewhere
\PassOptionsToPackage{unicode}{hyperref}
\PassOptionsToPackage{hyphens}{url}
\PassOptionsToPackage{dvipsnames,svgnames,x11names}{xcolor}
%
\documentclass[
  letterpaper,
  DIV=11,
  numbers=noendperiod]{scrartcl}
\usepackage{xcolor}
\usepackage[top=30mm,left=30mm]{geometry}
\usepackage{amsmath,amssymb}
\setcounter{secnumdepth}{-\maxdimen} % remove section numbering
\usepackage{iftex}
\ifPDFTeX
  \usepackage[T1]{fontenc}
  \usepackage[utf8]{inputenc}
  \usepackage{textcomp} % provide euro and other symbols
\else % if luatex or xetex
  \usepackage{unicode-math} % this also loads fontspec
  \defaultfontfeatures{Scale=MatchLowercase}
  \defaultfontfeatures[\rmfamily]{Ligatures=TeX,Scale=1}
\fi
\usepackage{lmodern}
\ifPDFTeX\else
  % xetex/luatex font selection
\fi
% Use upquote if available, for straight quotes in verbatim environments
\IfFileExists{upquote.sty}{\usepackage{upquote}}{}
\IfFileExists{microtype.sty}{% use microtype if available
  \usepackage[]{microtype}
  \UseMicrotypeSet[protrusion]{basicmath} % disable protrusion for tt fonts
}{}
\makeatletter
\@ifundefined{KOMAClassName}{% if non-KOMA class
  \IfFileExists{parskip.sty}{%
    \usepackage{parskip}
  }{% else
    \setlength{\parindent}{0pt}
    \setlength{\parskip}{6pt plus 2pt minus 1pt}}
}{% if KOMA class
  \KOMAoptions{parskip=half}}
\makeatother
% Make \paragraph and \subparagraph free-standing
\makeatletter
\ifx\paragraph\undefined\else
  \let\oldparagraph\paragraph
  \renewcommand{\paragraph}{
    \@ifstar
      \xxxParagraphStar
      \xxxParagraphNoStar
  }
  \newcommand{\xxxParagraphStar}[1]{\oldparagraph*{#1}\mbox{}}
  \newcommand{\xxxParagraphNoStar}[1]{\oldparagraph{#1}\mbox{}}
\fi
\ifx\subparagraph\undefined\else
  \let\oldsubparagraph\subparagraph
  \renewcommand{\subparagraph}{
    \@ifstar
      \xxxSubParagraphStar
      \xxxSubParagraphNoStar
  }
  \newcommand{\xxxSubParagraphStar}[1]{\oldsubparagraph*{#1}\mbox{}}
  \newcommand{\xxxSubParagraphNoStar}[1]{\oldsubparagraph{#1}\mbox{}}
\fi
\makeatother

\usepackage{color}
\usepackage{fancyvrb}
\newcommand{\VerbBar}{|}
\newcommand{\VERB}{\Verb[commandchars=\\\{\}]}
\DefineVerbatimEnvironment{Highlighting}{Verbatim}{commandchars=\\\{\}}
% Add ',fontsize=\small' for more characters per line
\usepackage{framed}
\definecolor{shadecolor}{RGB}{241,243,245}
\newenvironment{Shaded}{\begin{snugshade}}{\end{snugshade}}
\newcommand{\AlertTok}[1]{\textcolor[rgb]{0.68,0.00,0.00}{#1}}
\newcommand{\AnnotationTok}[1]{\textcolor[rgb]{0.37,0.37,0.37}{#1}}
\newcommand{\AttributeTok}[1]{\textcolor[rgb]{0.40,0.45,0.13}{#1}}
\newcommand{\BaseNTok}[1]{\textcolor[rgb]{0.68,0.00,0.00}{#1}}
\newcommand{\BuiltInTok}[1]{\textcolor[rgb]{0.00,0.23,0.31}{#1}}
\newcommand{\CharTok}[1]{\textcolor[rgb]{0.13,0.47,0.30}{#1}}
\newcommand{\CommentTok}[1]{\textcolor[rgb]{0.37,0.37,0.37}{#1}}
\newcommand{\CommentVarTok}[1]{\textcolor[rgb]{0.37,0.37,0.37}{\textit{#1}}}
\newcommand{\ConstantTok}[1]{\textcolor[rgb]{0.56,0.35,0.01}{#1}}
\newcommand{\ControlFlowTok}[1]{\textcolor[rgb]{0.00,0.23,0.31}{\textbf{#1}}}
\newcommand{\DataTypeTok}[1]{\textcolor[rgb]{0.68,0.00,0.00}{#1}}
\newcommand{\DecValTok}[1]{\textcolor[rgb]{0.68,0.00,0.00}{#1}}
\newcommand{\DocumentationTok}[1]{\textcolor[rgb]{0.37,0.37,0.37}{\textit{#1}}}
\newcommand{\ErrorTok}[1]{\textcolor[rgb]{0.68,0.00,0.00}{#1}}
\newcommand{\ExtensionTok}[1]{\textcolor[rgb]{0.00,0.23,0.31}{#1}}
\newcommand{\FloatTok}[1]{\textcolor[rgb]{0.68,0.00,0.00}{#1}}
\newcommand{\FunctionTok}[1]{\textcolor[rgb]{0.28,0.35,0.67}{#1}}
\newcommand{\ImportTok}[1]{\textcolor[rgb]{0.00,0.46,0.62}{#1}}
\newcommand{\InformationTok}[1]{\textcolor[rgb]{0.37,0.37,0.37}{#1}}
\newcommand{\KeywordTok}[1]{\textcolor[rgb]{0.00,0.23,0.31}{\textbf{#1}}}
\newcommand{\NormalTok}[1]{\textcolor[rgb]{0.00,0.23,0.31}{#1}}
\newcommand{\OperatorTok}[1]{\textcolor[rgb]{0.37,0.37,0.37}{#1}}
\newcommand{\OtherTok}[1]{\textcolor[rgb]{0.00,0.23,0.31}{#1}}
\newcommand{\PreprocessorTok}[1]{\textcolor[rgb]{0.68,0.00,0.00}{#1}}
\newcommand{\RegionMarkerTok}[1]{\textcolor[rgb]{0.00,0.23,0.31}{#1}}
\newcommand{\SpecialCharTok}[1]{\textcolor[rgb]{0.37,0.37,0.37}{#1}}
\newcommand{\SpecialStringTok}[1]{\textcolor[rgb]{0.13,0.47,0.30}{#1}}
\newcommand{\StringTok}[1]{\textcolor[rgb]{0.13,0.47,0.30}{#1}}
\newcommand{\VariableTok}[1]{\textcolor[rgb]{0.07,0.07,0.07}{#1}}
\newcommand{\VerbatimStringTok}[1]{\textcolor[rgb]{0.13,0.47,0.30}{#1}}
\newcommand{\WarningTok}[1]{\textcolor[rgb]{0.37,0.37,0.37}{\textit{#1}}}

\usepackage{longtable,booktabs,array}
\usepackage{calc} % for calculating minipage widths
% Correct order of tables after \paragraph or \subparagraph
\usepackage{etoolbox}
\makeatletter
\patchcmd\longtable{\par}{\if@noskipsec\mbox{}\fi\par}{}{}
\makeatother
% Allow footnotes in longtable head/foot
\IfFileExists{footnotehyper.sty}{\usepackage{footnotehyper}}{\usepackage{footnote}}
\makesavenoteenv{longtable}
\usepackage{graphicx}
\makeatletter
\newsavebox\pandoc@box
\newcommand*\pandocbounded[1]{% scales image to fit in text height/width
  \sbox\pandoc@box{#1}%
  \Gscale@div\@tempa{\textheight}{\dimexpr\ht\pandoc@box+\dp\pandoc@box\relax}%
  \Gscale@div\@tempb{\linewidth}{\wd\pandoc@box}%
  \ifdim\@tempb\p@<\@tempa\p@\let\@tempa\@tempb\fi% select the smaller of both
  \ifdim\@tempa\p@<\p@\scalebox{\@tempa}{\usebox\pandoc@box}%
  \else\usebox{\pandoc@box}%
  \fi%
}
% Set default figure placement to htbp
\def\fps@figure{htbp}
\makeatother

\ifLuaTeX
  \usepackage{luacolor}
  \usepackage[soul]{lua-ul}
\else
  \usepackage{soul}
\fi




\setlength{\emergencystretch}{3em} % prevent overfull lines

\providecommand{\tightlist}{%
  \setlength{\itemsep}{0pt}\setlength{\parskip}{0pt}}



 


\KOMAoption{captions}{tableheading}
\makeatletter
\@ifpackageloaded{tcolorbox}{}{\usepackage[skins,breakable]{tcolorbox}}
\@ifpackageloaded{fontawesome5}{}{\usepackage{fontawesome5}}
\definecolor{quarto-callout-color}{HTML}{909090}
\definecolor{quarto-callout-note-color}{HTML}{0758E5}
\definecolor{quarto-callout-important-color}{HTML}{CC1914}
\definecolor{quarto-callout-warning-color}{HTML}{EB9113}
\definecolor{quarto-callout-tip-color}{HTML}{00A047}
\definecolor{quarto-callout-caution-color}{HTML}{FC5300}
\definecolor{quarto-callout-color-frame}{HTML}{acacac}
\definecolor{quarto-callout-note-color-frame}{HTML}{4582ec}
\definecolor{quarto-callout-important-color-frame}{HTML}{d9534f}
\definecolor{quarto-callout-warning-color-frame}{HTML}{f0ad4e}
\definecolor{quarto-callout-tip-color-frame}{HTML}{02b875}
\definecolor{quarto-callout-caution-color-frame}{HTML}{fd7e14}
\makeatother
\makeatletter
\@ifpackageloaded{caption}{}{\usepackage{caption}}
\AtBeginDocument{%
\ifdefined\contentsname
  \renewcommand*\contentsname{Table of contents}
\else
  \newcommand\contentsname{Table of contents}
\fi
\ifdefined\listfigurename
  \renewcommand*\listfigurename{List of Figures}
\else
  \newcommand\listfigurename{List of Figures}
\fi
\ifdefined\listtablename
  \renewcommand*\listtablename{List of Tables}
\else
  \newcommand\listtablename{List of Tables}
\fi
\ifdefined\figurename
  \renewcommand*\figurename{Figure}
\else
  \newcommand\figurename{Figure}
\fi
\ifdefined\tablename
  \renewcommand*\tablename{Table}
\else
  \newcommand\tablename{Table}
\fi
}
\@ifpackageloaded{float}{}{\usepackage{float}}
\floatstyle{ruled}
\@ifundefined{c@chapter}{\newfloat{codelisting}{h}{lop}}{\newfloat{codelisting}{h}{lop}[chapter]}
\floatname{codelisting}{Listing}
\newcommand*\listoflistings{\listof{codelisting}{List of Listings}}
\makeatother
\makeatletter
\makeatother
\makeatletter
\@ifpackageloaded{caption}{}{\usepackage{caption}}
\@ifpackageloaded{subcaption}{}{\usepackage{subcaption}}
\makeatother
\usepackage{bookmark}
\IfFileExists{xurl.sty}{\usepackage{xurl}}{} % add URL line breaks if available
\urlstyle{same}
\hypersetup{
  pdftitle={Introduction to Literate Programming with Quarto},
  pdfauthor={Elizabeth Waterfield},
  colorlinks=true,
  linkcolor={blue},
  filecolor={Maroon},
  citecolor={Blue},
  urlcolor={Blue},
  pdfcreator={LaTeX via pandoc}}


\title{Introduction to Literate Programming with Quarto}
\author{Elizabeth Waterfield}
\date{20/08/2025}
\begin{document}
\maketitle


\subsection{Credit statement and
licence}\label{credit-statement-and-licence}

ttt-quarto-slides

Possible roles using the CRediT contribition system:

\begin{itemize}
\item
  \textbf{Conceptualization}: Ideas; formulation or evolution of
  overarching research goals and aims
\item
  \textbf{Methodology}: Development or design of methodology; creation
  of models
\item
  \textbf{Software} : Programming, software development; designing
  computer programs; implementation of the computer code and supporting
  algorithms; testing of existing code components
\item
  \textbf{Validation}: Verification, whether as a part of the activity
  or separate, of the overall replication/ reproducibility of
  results/experiments and other research outputs
\item
  \textbf{Formal analysis}: Application of statistical, mathematical,
  computational, or other formal techniques to analyze or synthesize
  study data
\item
  \textbf{Investigation}: Conducting a research and investigation
  process, specifically performing the experiments, or data/evidence
  collection
\item
  \textbf{Resources}: Provision of study materials, reagents, materials,
  patients, laboratory samples, animals, instrumentation, computing
  resources, or other analysis tools
\item
  \textbf{Data Curation}: Management activities to annotate (produce
  metadata), scrub data and maintain research data (including software
  code, where it is necessary for interpreting the data itself) for
  initial use and later reuse
\item
  \textbf{Writing - Original Draft}: Preparation, creation and/or
  presentation of the published work, specifically writing the initial
  draft (including substantive translation)
\item
  \textbf{Writing - Review \& Editing}: Preparation, creation and/or
  presentation of the published work by those from the original research
  group, specifically critical review, commentary or revision --
  including pre-or postpublication stages
\item
  \textbf{Visualization}: Preparation, creation and/or presentation of
  the published work, specifically visualization/ data presentation
\item
  \textbf{Supervision}: Oversight and leadership responsibility for the
  research activity planning and execution, including mentorship
  external to the core team
\item
  \textbf{Project administration}: Management and coordination
  responsibility for the research activity planning and execution
\item
  \textbf{Funding acquisition}: Acquisition of the financial support for
  the project leading to this publication
\end{itemize}

\begin{center}\rule{0.5\linewidth}{0.5pt}\end{center}

\subsubsection{Prerequisites}\label{prerequisites}

\begin{tcolorbox}[enhanced jigsaw, colback=white, titlerule=0mm, toprule=.15mm, breakable, colframe=quarto-callout-important-color-frame, toptitle=1mm, opacityback=0, title=\textcolor{quarto-callout-important-color}{\faExclamation}\hspace{0.5em}{Prerequisites}, coltitle=black, colbacktitle=quarto-callout-important-color!10!white, leftrule=.75mm, opacitybacktitle=0.6, bottomrule=.15mm, bottomtitle=1mm, arc=.35mm, rightrule=.15mm, left=2mm]

Before completing this submodule, please carefully read about the
necessary prerequisites.

\end{tcolorbox}

\begin{longtable}[]{@{}
  >{\raggedright\arraybackslash}p{(\linewidth - 4\tabcolsep) * \real{0.3333}}
  >{\raggedright\arraybackslash}p{(\linewidth - 4\tabcolsep) * \real{0.3333}}
  >{\raggedright\arraybackslash}p{(\linewidth - 4\tabcolsep) * \real{0.3333}}@{}}
\toprule\noalign{}
\begin{minipage}[b]{\linewidth}\raggedright
Prerequisite
\end{minipage} & \begin{minipage}[b]{\linewidth}\raggedright
Description
\end{minipage} & \begin{minipage}[b]{\linewidth}\raggedright
Link/Where to find it
\end{minipage} \\
\midrule\noalign{}
\endhead
\bottomrule\noalign{}
\endlastfoot
Topic Name & Basic intro to X & Module + Submodule \\
Software Name & Configuring the environment &
\href{https://quarto.org}{Download Link} \\
\end{longtable}

\emph{Speaker Notes} Speaker notes may act as a guiding script for
delivering the presentation. These notes contain key information, such
as phrases, explanations, or transitions, that can be said aloud to
clarify concepts, emphasize important points, or maintain the flow of
information.

\emph{Instructor Notes} Instructor notes provide teaching support but
are not intended to be spoken directly. These notes contain additional
context, such as learning objectives, common issues, and pedagogical
tips, to help the instructor adapt their teaching to the learner's needs
as well as anticipate challenges.

\begin{center}\rule{0.5\linewidth}{0.5pt}\end{center}

\subsection{Questions from previous
submodule}\label{questions-from-previous-submodule}

\emph{Instructor Notes} - Aim: clarify questions from the previous
submodule and/or to discuss assignments. - Additional slides may need to
be added depending on the nature of the homework assignments. - It is
critical for the learning process to ensure that students are on the
same page and have been able to achieve the learning goals of the
previous workshop. - Not applicable if this set of slides corresponds to
the first submodule of a new module.

\subsection{Before we start: Survey
time!}\label{before-we-start-survey-time}

Take this survey to test your prior Quarto knowledge

Quarto Survey link

\emph{Instructor Notes} - Aim: the pre-submodule survey serves to
examine students' prior knowledge about the sumodule's topic. - Use free
survey software such as or other survey software (Menti, particify,
formR) to establish this. You can use the example survey, edit it or
create your own.

\begin{center}\rule{0.5\linewidth}{0.5pt}\end{center}

\subsection{Discussion of survey
results}\label{discussion-of-survey-results}

\emph{Instructor Notes} - Aim: briefly examine the answers given to each
question interactively with the group. - Use visuals from the survey to
highlight specific answers. - Make it clear to the group that there will
be a similar post-submodule survey to examine understanding and learning
progress.

\begin{center}\rule{0.5\linewidth}{0.5pt}\end{center}

\subsection{Where are we at?}\label{where-are-we-at}

\emph{Instructor Notes} - Aim: Place the topic of the current submodule
within a broader context. - Remind students what you are working towards
and what the bigger picture is.

\begin{center}\rule{0.5\linewidth}{0.5pt}\end{center}

\subsection{Quarto}\label{quarto}

\textbf{Quarto} makes it easy to analyze, share and reproduce.

It's a powerful tool for producing transparent, reproducible, and
accessible work that can be freely shared, viewed, and reused by others.

\begin{center}\rule{0.5\linewidth}{0.5pt}\end{center}

\subsection{Quarto Learning Goals}\label{quarto-learning-goals}

\begin{itemize}
\tightlist
\item
  Participants will \textbf{learn} how to create, edit, and render
  Quarto documents
\item
  Participants will \textbf{understand} how to use key Quarto features
  (code chunks, YAML headers, citations, and output formatting)
\item
  Participants will \textbf{gain confidence} in preparing and teaching
  Quarto effectively to students
\end{itemize}

\emph{Instructor Notes} - Aim: Formulate specific, action-oriented goals
learning goals which are measurable and observable in line with Bloom's
taxonomy (Anderson et al., 2001; Bloom et al., 1956) - Emphasis is
placed on the \textbf{verbs} of the learning goals- choose verbs that
align with the skills you want to develop or assess.

\begin{itemize}
\item
  Examples:

  \begin{itemize}
  \tightlist
  \item
    Students will \textbf{describe} the process of photosynthesis or
  \item
    Students will \textbf{construct} a diagram illustrating the process
    of photosynthesis
  \end{itemize}
\end{itemize}

\begin{center}\rule{0.5\linewidth}{0.5pt}\end{center}

\subsection{Key Terms and Definitions}\label{key-terms-and-definitions}

\textbf{What is Quarto?} An open-source scientific and publishing system
that combines text, code and media to create easily shareable documents
such as websites, slides, reports, and much more.

\textbf{What is Rendering?} The process where Quarto runs the code,
combines it with the text, and creates a final output.

A Quarto Markdown document is saved as a \textbf{.qmd file}

\emph{Instructor Notes} - Aim: Introduce key terms and definitions that
students will come across throughout the session. - This first part of
the lesson is useful to establish an understanding of important
vocabulary- it can be helpful to remind students of the meaning of these
terms as they appear in the upcoming sections.

\begin{center}\rule{0.5\linewidth}{0.5pt}\end{center}

\subsection{Key Terms and
Definitions}\label{key-terms-and-definitions-1}

\textbf{Components of a .qmd file}

\begin{itemize}
\item
  \textbf{YAML header}- the section at the top of the Quarto document
  that controls settings like the title, output format, and author.
\item
  \textbf{code chunks}- sections of the document that contains code
  (from R or Python, for example) that are used for showing results such
  as tables, plots, or calculations
\item
  \textbf{Quarto Markdown}- combines text, codes, and formatting to
  create the actual content of the document
\end{itemize}

\begin{center}\rule{0.5\linewidth}{0.5pt}\end{center}

\subsection{Key Terms and
Definitions}\label{key-terms-and-definitions-2}

\textbf{Components of a .qmd file}

You can write Quarto documents in \textbf{Source mode} or \textbf{Visual
mode} in RStudio.

\begin{itemize}
\item
  \textbf{Source mode} - you write directly in plain text/Markdown
  Syntax, allowing for more control and it's closer to raw code
\item
  \textbf{Visual mode} - gives a WYSIWYM-style interface which easier
  for beginners, but sometimes hides syntax.
\item
  This is part of the \textbf{authoring process}, and it allows you to
  format the text, add code, and build your document.
\end{itemize}

\emph{Instructor Notes} Base yourself on conceptual change theory and
examine existing concepts in relation to some key terms. Re-examine
formation of new concepts at the end of the lesson.

\emph{Speaker Notes} - WYSIWYM is an acronym for ``What You Say is What
You Mean'' which means that whatever you see in your document will be
what you see even when you save or render the document as PDF or as
HTML.

\begin{itemize}
\tightlist
\item
  Quarto documents can be created and edited in either Source mode or
  Visual mode. Both modes work with the same file, but they provide
  slightly different experiences. In this lesson, most of our tasks will
  be done in Source mode, but you'll also have a chance to try Visual
  mode so you can compare how the same content looks and feels in each.
\end{itemize}

\begin{center}\rule{0.5\linewidth}{0.5pt}\end{center}

\subsection{Covered in this Session}\label{covered-in-this-session}

\begin{itemize}
\tightlist
\item
  Setting up a Quarto Document
\item
  Authoring
\item
  Code Chunks
\item
  Adding Citations
\item
  Publishing
\end{itemize}

\emph{Instructor Notes} - Aim: Core theoretical introduction of
submodule topic. - Pair theoretical aspects with practical exercises and
group discussions according to the Think-Pair-Share style and according
to Cognitive Load Theory (Sweller, 1980).

\begin{itemize}
\tightlist
\item
  For a 90-minute lesson, the instructor should try to ``lecture'' for
  only 20 minutes, students should work in groups/pairs/on their own for
  at least 55 minutes of the lesson (+ a 15 minute break).
\end{itemize}

Practical exercises on topic - Aim: practical exercises for students to
apply the new skills in practise. Each submodule topic will include
corresponding Tasks- adopting a ``learn by doing'' approach. - Depending
on the topic, the exercises should be in accordance with the learning
objective(s). - It's useful to have exercises directly after a topic is
taught to reinforce what was learnt.

For students who advance faster: Prepare extra exercises.

\begin{center}\rule{0.5\linewidth}{0.5pt}\end{center}

\subsection{Setting up a Quarto Document using
RStudio}\label{setting-up-a-quarto-document-using-rstudio}

\pandocbounded{\includegraphics[keepaspectratio]{images/setup1.png}}

\begin{itemize}
\tightlist
\item
  Select \textbf{File}\\
\item
  → Select \textbf{New File}\\
\item
  → Select \textbf{Quarto Document}
\end{itemize}

\begin{center}\rule{0.5\linewidth}{0.5pt}\end{center}

\subsection{Setting up a Quarto Document using
RStudio}\label{setting-up-a-quarto-document-using-rstudio-1}

\pandocbounded{\includegraphics[keepaspectratio]{images/setup2.png}}

\begin{itemize}
\tightlist
\item
  The default format is html
\item
  The output format can also be changed by editing the YAML header
\end{itemize}

\textbf{Task 1}

\begin{itemize}
\tightlist
\item[$\square$]
  Follow the steps to create a new Quarto Document
\end{itemize}

\emph{Instructor Notes} - Ensure students know where their Quarto
project folder is located on their device. This will make it easier for
them to find files in later tasks. Encourage them to save the document
right after creating it, so they can intentionally choose the folder
location (e.g., Desktop, a dedicated course folder, etc.).

\begin{center}\rule{0.5\linewidth}{0.5pt}\end{center}

\subsection{Rendering}\label{rendering}

There are two ways to render in Quarto

\begin{itemize}
\tightlist
\item
  \textbf{Manual rendering}: having to click on the ``Render'' each time
  you want to see the output
\item
  \textbf{Render on Save}: Quarto will automatically re-render the
  document each time you click ``Save''
\end{itemize}

\begin{figure}[H]

{\centering \pandocbounded{\includegraphics[keepaspectratio]{images/rendering.png}}

}

\caption{You can find both Manual Render and Render on Save at the top
of your workspace}

\end{figure}%

\emph{Speaker Notes} - Quarto documents can be rendered either manually
or automatically when you save. - Manual rendering gives you control
meaning you decide when to update the document, which is helpful if
you're making lots of edits and don't want constant re-renders. - Render
on Save automatically updates your document every time you hit Save,
which is great for quick feedback and seeing your changes instantly. In
this lesson, you can try both to see which workflow feels more
comfortable for you.

\begin{center}\rule{0.5\linewidth}{0.5pt}\end{center}

\subsection{Authoring}\label{authoring}

This is the process of writing and structuring the Quarto document.

\textbf{YAML Header + Markdown = Authoring}

To practice authoring in Quarto, let's start with setting the YAML
header and adding text in \textbf{Source} mode.

\emph{Instructor Notes} - Reminding students on the meanings of YAML
Header and Markdown will be helpful especially to those with very little
prior knowledge for working with Quarto.

\emph{Speaker Notes} - The YAML header is the section at the very top,
wrapped in three dashes (---). It holds the document's settings, like
the title, author, date, and output format. - Below that, we use
Markdown to write the actual content which are things like headings,
bullet points, bold or italic text.

\begin{center}\rule{0.5\linewidth}{0.5pt}\end{center}

\subsection{Authoring}\label{authoring-1}

\textbf{Task 2}

\begin{itemize}
\tightlist
\item[$\square$]
  Copy \& paste the following into the YAML Header of your document
\end{itemize}

\begin{Shaded}
\begin{Highlighting}[]
\PreprocessorTok{{-}{-}{-}}
\FunctionTok{title}\KeywordTok{:}\AttributeTok{ }\StringTok{"ChickWeight Analysis"}
\FunctionTok{author}\KeywordTok{:}\AttributeTok{ }\StringTok{"Your Name"}
\FunctionTok{format}\KeywordTok{:}
\AttributeTok{  }\FunctionTok{html}\KeywordTok{:}
\AttributeTok{    }\FunctionTok{code{-}fold}\KeywordTok{:}\AttributeTok{ }\CharTok{false}
\AttributeTok{    }\FunctionTok{toc}\KeywordTok{:}\AttributeTok{ }\CharTok{true}
\PreprocessorTok{{-}{-}{-}}
\end{Highlighting}
\end{Shaded}

\begin{tcolorbox}[enhanced jigsaw, colback=white, titlerule=0mm, toprule=.15mm, breakable, colframe=quarto-callout-note-color-frame, toptitle=1mm, opacityback=0, title=\textcolor{quarto-callout-note-color}{\faInfo}\hspace{0.5em}{output format}, coltitle=black, colbacktitle=quarto-callout-note-color!10!white, leftrule=.75mm, opacitybacktitle=0.6, bottomrule=.15mm, bottomtitle=1mm, arc=.35mm, rightrule=.15mm, left=2mm]

You can replace ``html'' to render the document to a different format.

Here is a link to a list of the different output formats.

\end{tcolorbox}

\emph{Instructor Note} - There is a small clipboard icon at the
top-right of the text box. Let students know they can simply click the
icon to copy all the text in that specific text box. This icon will
appear for every text box moving forward so you can remind them of it
the next few times a task requires them to copy the content of a text
box.

\emph{Speaker Note} - Remember that the YAML Header goes at the top of
the document in both Source and Visual Modes.

\begin{itemize}
\tightlist
\item
  In the YAML header, the format field tells Quarto what kind of output
  to create. Here we've set it to html, which means the document will
  render as a web page. You could choose other formats like PDF or Word,
  but for this lesson we'll stick with HTML so everyone has the same
  experience.
\end{itemize}

\begin{center}\rule{0.5\linewidth}{0.5pt}\end{center}

\subsection{Authoring}\label{authoring-2}

\textbf{Basic Markdown Text Formatting}

\begin{itemize}
\item
  \textbf{Bold:} \texttt{**bold**} → \textbf{bold}
\item
  \emph{Italic:} \texttt{*italic*} → \emph{italic}
\item
  \st{Strikethrough:}
  \texttt{\textasciitilde{}\textasciitilde{}text\textasciitilde{}\textasciitilde{}}
  → \st{text}
\item
  \texttt{Inline\ code:} \texttt{\textasciigrave{}code\textasciigrave{}}
  → \texttt{code}
\end{itemize}

\begin{tcolorbox}[enhanced jigsaw, colback=white, titlerule=0mm, toprule=.15mm, breakable, colframe=quarto-callout-note-color-frame, toptitle=1mm, opacityback=0, title=\textcolor{quarto-callout-note-color}{\faInfo}\hspace{0.5em}{markdown shortcuts}, coltitle=black, colbacktitle=quarto-callout-note-color!10!white, leftrule=.75mm, opacitybacktitle=0.6, bottomrule=.15mm, bottomtitle=1mm, arc=.35mm, rightrule=.15mm, left=2mm]

Here is a link for the full list of markdown shortcuts for formatting!

\end{tcolorbox}

\emph{Speaker Note} - In Markdown, we can format text with very simple
symbols. It works the same way in both Source and Visual mode with the
only difference being how you see it while typing. - In Source mode you
see the symbols, while in Visual mode it looks like regular bold or
italic text right away.

\begin{center}\rule{0.5\linewidth}{0.5pt}\end{center}

\subsection{Authoring}\label{authoring-3}

\begin{Shaded}
\begin{Highlighting}[]
\NormalTok{Have you ever been curious about what affects a chick\textquotesingle{}s weight? }
\NormalTok{This document explores the ChickWeight dataset using R.  }
\NormalTok{The goal is to compare chick weight across different diets and time points.  }
\NormalTok{Key steps include:}

\SpecialStringTok{{-} }\NormalTok{Loading the dataset  }
\SpecialStringTok{{-} }\NormalTok{Visualizing growth trends  }
\SpecialStringTok{{-} }\NormalTok{Summarizing results}
\end{Highlighting}
\end{Shaded}

\textbf{Task 3}

\begin{itemize}
\tightlist
\item[$\square$]
  Copy the Markdown Text and paste it to your document
\item[$\square$]
  Bold the word \emph{``ChickWeight''}
\item[$\square$]
  Italicize the phrase \emph{``growth trends''}
\end{itemize}

\begin{center}\rule{0.5\linewidth}{0.5pt}\end{center}

\subsection{Authoring}\label{authoring-4}

Highlight particularly important aspects with \textbf{Quarto callout
boxes}

\begin{tcolorbox}[enhanced jigsaw, colback=white, titlerule=0mm, toprule=.15mm, breakable, colframe=quarto-callout-important-color-frame, toptitle=1mm, opacityback=0, title=\textcolor{quarto-callout-important-color}{\faExclamation}\hspace{0.5em}{Important with Title}, coltitle=black, colbacktitle=quarto-callout-important-color!10!white, leftrule=.75mm, opacitybacktitle=0.6, bottomrule=.15mm, bottomtitle=1mm, arc=.35mm, rightrule=.15mm, left=2mm]

This is an example of a callout box to highlight particularly important
information using \texttt{callout-important}

\end{tcolorbox}

\begin{tcolorbox}[enhanced jigsaw, colback=white, titlerule=0mm, toprule=.15mm, breakable, colframe=quarto-callout-tip-color-frame, toptitle=1mm, opacityback=0, title=\textcolor{quarto-callout-tip-color}{\faLightbulb}\hspace{0.5em}{Tip with Title}, coltitle=black, colbacktitle=quarto-callout-tip-color!10!white, leftrule=.75mm, opacitybacktitle=0.6, bottomrule=.15mm, bottomtitle=1mm, arc=.35mm, rightrule=.15mm, left=2mm]

This is an example of a callout box to give important tips using
\texttt{callout-tip}

\end{tcolorbox}

\begin{tcolorbox}[enhanced jigsaw, colback=white, titlerule=0mm, toprule=.15mm, breakable, colframe=quarto-callout-note-color-frame, toptitle=1mm, opacityback=0, title=\textcolor{quarto-callout-note-color}{\faInfo}\hspace{0.5em}{Note with Title}, coltitle=black, colbacktitle=quarto-callout-note-color!10!white, leftrule=.75mm, opacitybacktitle=0.6, bottomrule=.15mm, bottomtitle=1mm, arc=.35mm, rightrule=.15mm, left=2mm]

This is an example of a callout box to include an additional note using
\texttt{callout-note}

\end{tcolorbox}

\emph{Instructor Note} - These are just examples of callout boxes. How
to actually create one will be covered in the next slide/task.

\emph{Speaker Note} - Callout boxes are a way to highlight important
information in your document. They stand out visually, so readers
immediately notice them. You can use them to emphasize tips, warnings,
examples, or key takeaways.

\begin{center}\rule{0.5\linewidth}{0.5pt}\end{center}

\subsection{Authoring}\label{authoring-5}

\textbf{Adding a Callout Box}

Here is the markdown text for inserting a \textbf{callout note} box.

\begin{Shaded}
\begin{Highlighting}[]
\NormalTok{::: callout{-}note}
\FunctionTok{\#\# Based on Real Data}

\NormalTok{The ChickWeight dataset in R is based on real experimental data.}
\NormalTok{:::}
\end{Highlighting}
\end{Shaded}

\textbf{Task 4}

\begin{itemize}
\tightlist
\item[$\square$]
  Copy \& paste into your Quarto document to add this callout note box.
\end{itemize}

\begin{center}\rule{0.5\linewidth}{0.5pt}\end{center}

\subsection{Code Chunks}\label{code-chunks}

\textbf{Inserting Code}

Two ways to insert code chunks:

\begin{itemize}
\tightlist
\item
  Manually type 3 back ticks ``` then \{r\} to start a coding chunk,
  enter your code, then end it with 3 back ticks
\end{itemize}

\textbf{OR}

\begin{itemize}
\tightlist
\item
  Use the keyboard shortcut Ctrl + Alt + I (Windows/Linux) or Cmd +
  Option + I (Mac) to insert a code chunk
\end{itemize}

\emph{Instructor Notes} - using \{r\} to start the Code Chunk indicates
that the coding language that we're using is R. This is relevant for
this lesson as students will be using R Studio and possession pof some
background knowledge in R is assumed.

\emph{Speaker Notes} - Code chunks are sections of a Quarto document
where we can run code directly inside our file. They're marked by three
backticks followed by the language name, like \{r\} or \{python\}. -
These let us write and execute code, and then display the results, such
as tables, plots, or calculations, right in the document. Code chunks
make it possible to combine text and analysis in one place, so the
document stays reproducible and dynamic.

\begin{center}\rule{0.5\linewidth}{0.5pt}\end{center}

\subsection{Code Chunks}\label{code-chunks-1}

\textbf{Inserting Code}

\begin{Shaded}
\begin{Highlighting}[]
\NormalTok{summary(ChickWeight)}
\end{Highlighting}
\end{Shaded}

\begin{Shaded}
\begin{Highlighting}[]
\NormalTok{library(ggplot2)}
\NormalTok{ggplot(ChickWeight, aes(x = Time, y = weight, color = Diet)) +}
\NormalTok{  geom\_line(aes(group = Chick)) +}
\NormalTok{  labs(title = "Chick Growth Over Time")}
\end{Highlighting}
\end{Shaded}

\textbf{Task 5}

\begin{itemize}
\tightlist
\item[$\square$]
  Insert the code in \textbf{two separate code chunks}
\item[$\square$]
  Render the output and see what you get
\end{itemize}

\begin{center}\rule{0.5\linewidth}{0.5pt}\end{center}

\subsection{Code Chunks}\label{code-chunks-2}

Adjust how the code is portrayed by editing the YAML header

\pandocbounded{\includegraphics[keepaspectratio]{images/exercise1.png}}

\begin{itemize}
\tightlist
\item
  \texttt{code-fold:\ false} - the code is visible and not collapsible
  (as seen here)
\item
  \texttt{code-fold:\ true} - collapses the code so the reader can
  expand it
\end{itemize}

\emph{Speaker Notes} - Quarto gives us several options for controlling
how code appears in our documents. For example, we can choose whether to
show or hide the code itself, whether readers can fold code open and
closed, or whether only the results are shown. - Currently in our YAML,
we have it set to \texttt{code-fold:\ false} which means the code is
displayed as in the image. - These settings don't change the analysis,
but just change how it's displayed. This flexibility is helpful because
sometimes we want to highlight the process by showing the code, and
other times we want readers to focus on the results. - In open research,
showing the underlying code isn't just a technical choice, it's
important for building trust and reproducibility.

\begin{center}\rule{0.5\linewidth}{0.5pt}\end{center}

\subsection{Code Chunks}\label{code-chunks-3}

Adjust how the code is portrayed by editing the YAML header

\pandocbounded{\includegraphics[keepaspectratio]{images/exercise1.png}}

\begin{itemize}
\tightlist
\item
  \texttt{code-tools:\ true} - adds the functions ``show code'' at the
  top of the page and ``copy'' next to the chunks
\item
  \texttt{echo:\ true} - both the code and the output is visible
\end{itemize}

\emph{Instructor Notes} It's easy to confuse \texttt{echo} and
\texttt{code-fold} because they both deal with code visibility. Be clear
on the difference

\emph{Speaker Notes} - \texttt{echo:\ true} and
\texttt{code-fold:\ false} both affect how code is shown, but in
different ways. \texttt{echo} decides if the code prints at all, while
\texttt{code-fold} lets readers toggle it open or closed, deciding
whether the code can be hidden or expanded.

\begin{center}\rule{0.5\linewidth}{0.5pt}\end{center}

\subsection{Code Chunks}\label{code-chunks-4}

Let's edit the YAML header to make the code chunks \textbf{collapsible}
and add \textbf{code tools}.

\textbf{Task 6}

\begin{itemize}
\tightlist
\item[$\square$]
  In the YAML, change \texttt{code-fold:\ false} to
  \texttt{code-fold:\ true}
\item[$\square$]
  Add \texttt{code-tools:\ true}
\item[$\square$]
  Render the document to see the changes
\end{itemize}

\begin{center}\rule{0.5\linewidth}{0.5pt}\end{center}

\subsection{Pre-break survey}\label{pre-break-survey}

\emph{Instructor Note} - Aim: This pre-break survey serves to examine
students' current understanding of key concepts of the submodule - Use
free survey software such as or other survey software (Menti, particify,
formR) to establish the following questions.

\textbf{What's the name of Quarto Markdown document}?

\begin{enumerate}
\def\labelenumi{\alph{enumi}.}
\item
  .png file
\item
  .qmd file ✅
\item
  .doc file
\item
  .mp3 file
\end{enumerate}

\textbf{What is the YAML header?}

\begin{enumerate}
\def\labelenumi{\alph{enumi}.}
\item
  summarizes the document into a single line
\item
  this is where you store notes and reminders
\item
  it's like the ``settings'' of the document ✅
\item
  just the title of the document
\end{enumerate}

\textbf{Which of the following is used to format content (like
paragraphs and bullet points)?}

\begin{enumerate}
\def\labelenumi{\alph{enumi}.}
\item
  Code chunks
\item
  YAML header
\item
  Markdown text ✅
\item
  All of the above
\end{enumerate}

\textbf{What are the 2 components of the Authoring process?}

\textbf{How to create a code chunk in a Quarto Document?}

\section{Break! 10 minutes}\label{break-10-minutes}

\begin{center}\rule{0.5\linewidth}{0.5pt}\end{center}

\subsection{Post-break survey
discussion}\label{post-break-survey-discussion}

\emph{Instructor Note} - Aim: To clarify concepts and aspects that are
not yet understood - Highlight specific answers given during the survey

\begin{center}\rule{0.5\linewidth}{0.5pt}\end{center}

\subsection{Additional Authoring
Features}\label{additional-authoring-features}

Quarto offers additional \textbf{authoring features} that make it more
versatile and comprehensive. These include:

\begin{itemize}
\tightlist
\item
  adding links and hyperlinking text,
\item
  embedding media,
\item
  creating multi-column layouts, and more.
\end{itemize}

\begin{tcolorbox}[enhanced jigsaw, colback=white, titlerule=0mm, toprule=.15mm, breakable, colframe=quarto-callout-note-color-frame, toptitle=1mm, opacityback=0, title=\textcolor{quarto-callout-note-color}{\faInfo}\hspace{0.5em}{commonly used authoring features}, coltitle=black, colbacktitle=quarto-callout-note-color!10!white, leftrule=.75mm, opacitybacktitle=0.6, bottomrule=.15mm, bottomtitle=1mm, arc=.35mm, rightrule=.15mm, left=2mm]

Here is a link for commonly used markdown syntax for additional
authoring features!

\end{tcolorbox}

\emph{Speaker Note} - Beyond writing plain text, Quarto also allows us
to add features like hyperlinks, images or videos, and multi-column
layouts. These improve how our document communicates. - Hyperlinks let
us connect to sources or other sections, media like images and video
help explain complex ideas more clearly, and multi-column layouts make
the content easier to read and more visually appealing. - Using these
features thoughtfully improves both the usability and the overall look
of the document.

\begin{center}\rule{0.5\linewidth}{0.5pt}\end{center}

\subsection{Inserting Links}\label{inserting-links}

link with title: {[}title{]} (link)

\begin{itemize}
\tightlist
\item
  {[} Quarto Website {]} ( https://quarto.org/ ) →
  \href{https://quarto.org/}{Quarto Website}
\item
  {[}Reveal.js Documentation{]} (https://revealjs.com/) →
  \href{https://revealjs.com/}{Reveal.js Documentation}
\end{itemize}

link without title: \textless{} https:// link \textgreater{}

\begin{itemize}
\tightlist
\item
  \textless{} https://www.markdownguide.org/ \textgreater{} →
  \url{https://www.markdownguide.org/}
\item
  \textless{} https://github.com/ \textgreater{} →
  \url{https://github.com/}
\end{itemize}

\begin{center}\rule{0.5\linewidth}{0.5pt}\end{center}

\subsection{Inserting Images}\label{inserting-images}

\begin{itemize}
\tightlist
\item
  no caption: ! {[} {]} ( path/image.png )
\end{itemize}

\includegraphics[width=0.2\linewidth,height=\textheight,keepaspectratio]{images/chickpic1.png}

\emph{Speaker Note} - A benefit of using Quarto is the ability to easily
add media, such as images, to your documents

\begin{center}\rule{0.5\linewidth}{0.5pt}\end{center}

\subsection{Inserting Images}\label{inserting-images-1}

\begin{itemize}
\tightlist
\item
  with caption: ! {[} caption {]} ( path/image.png )
\end{itemize}

\begin{figure}[H]

{\centering \includegraphics[width=0.2\linewidth,height=\textheight,keepaspectratio]{images/chickpic1.png}

}

\caption{This is a caption about three yellow chicks in the grass.}

\end{figure}%

\begin{center}\rule{0.5\linewidth}{0.5pt}\end{center}

\subsection{Inserting Images}\label{inserting-images-2}

\begin{itemize}
\tightlist
\item
  with link: {[} ! {[} caption {]} ( path/image.png ) {]} ( link )
\end{itemize}

\begin{figure}[H]

{\centering \includegraphics[width=0.65\linewidth,height=\textheight,keepaspectratio]{images/chickpic1.png}

}

\caption{Click to see a study on broiler chicks}

\end{figure}%

\begin{center}\rule{0.5\linewidth}{0.5pt}\end{center}

\subsection{Inserting Images}\label{inserting-images-3}

\begin{Shaded}
\begin{Highlighting}[]
\AlertTok{![No chicks were harmed in the making of this lesson.](images/chickpic1.png)}
\end{Highlighting}
\end{Shaded}

\textbf{Task 7}

\begin{itemize}
\tightlist
\item[$\square$]
  Click this link to download the image
\item[$\square$]
  Save the image to the main folder of the Quarto document files
\item[$\square$]
  Copy \& paste the markdown text to insert an image with a caption
  (edit path if needed)
\end{itemize}

\emph{Instructor Note} - Setting the path can be tricky depending on
where the files are located - It's a good practice to save these files
in the same main folder as your Quarto document files (ideally in a
dedicated subfolder named ``images'' to keep paths organized and easy to
manage). If students saved and placed their main folder earlier, then
locating it should be easy. - in the Task above, the text says
``images/chickpic1.png'' assuming the picture is saved in a folder
called ``Images'' in the main folder containing all the Quarto document
files.

\begin{center}\rule{0.5\linewidth}{0.5pt}\end{center}

\subsection{Creating Columns}\label{creating-columns}

You can change the layout of a section using \textbf{columns}.

\begin{itemize}
\tightlist
\item
  Start the column layout by writing ``:::::: columns''
\item
  Begin the first column with ``::: column''
\item
  Add content and end it with ``:::''
\item
  Begin second coloumn with ``::: column''
\item
  Add content and end it with ``:::''
\item
  End section with a final ``::::::''
\end{itemize}

\begin{tcolorbox}[enhanced jigsaw, colback=white, titlerule=0mm, toprule=.15mm, breakable, colframe=quarto-callout-note-color-frame, toptitle=1mm, opacityback=0, title=\textcolor{quarto-callout-note-color}{\faInfo}\hspace{0.5em}{list of page layout options}, coltitle=black, colbacktitle=quarto-callout-note-color!10!white, leftrule=.75mm, opacitybacktitle=0.6, bottomrule=.15mm, bottomtitle=1mm, arc=.35mm, rightrule=.15mm, left=2mm]

Here is a link for page layout options to author how content looks on
your document!

\end{tcolorbox}

\emph{Speaker Note} - Columns are useful when you want to organize
content side by side instead of in a long vertical flow. For e.g., you
might place an image in one column and an explanation in the other, or
compare two code outputs directly next to each other like comparing two
separate plots. - Columns help make information easier to scan and can
improve the balance and readability of your slides or documents.

\begin{center}\rule{0.5\linewidth}{0.5pt}\end{center}

\subsection{Creating Columns}\label{creating-columns-1}

Let's use columns to add bullet points next to an image.

\textbf{Task 8}

\begin{itemize}
\tightlist
\item[$\square$]
  Click this link to download the image
\item[$\square$]
  Save the image to the main folder of the Quarto document files
\end{itemize}

\begin{center}\rule{0.5\linewidth}{0.5pt}\end{center}

\subsection{Creating Columns}\label{creating-columns-2}

\textbf{Task 8}

\begin{Shaded}
\begin{Highlighting}[]
\NormalTok{:::::: columns}
\NormalTok{::: column}

\AlertTok{![](images/chickpic2.png)}

\NormalTok{:::}

\NormalTok{::: column}

\SpecialStringTok{{-}   }\NormalTok{Monitor growth trends over time}
\SpecialStringTok{{-}   }\NormalTok{Compare diets and weight gain}

\NormalTok{:::}
\NormalTok{::::::}
\end{Highlighting}
\end{Shaded}

\begin{itemize}
\tightlist
\item[$\square$]
  Copy \& paste the markdown text to add an image with bullet points
  using columns (edit path if needed)
\end{itemize}

\begin{center}\rule{0.5\linewidth}{0.5pt}\end{center}

\subsection{Adding Citations with
Zotero}\label{adding-citations-with-zotero}

\textbf{Zotero} is a free reference management tool to collect,
organize, cite, and share research sources.

You can use \textbf{Zotero} in RStudio to easily insert citations into
your Quarto document

\emph{Speaker Note} - Zotero is a reference manager that helps you
collect and organize research articles, books, and other sources. When
you connect Zotero to Quarto, you can easily insert citations while
writing and then automatically generate a reference list at the end.
This saves time and reduces errors compared to typing citations
manually.

\begin{center}\rule{0.5\linewidth}{0.5pt}\end{center}

\subsection{Adding Citations with
Zotero}\label{adding-citations-with-zotero-1}

\pandocbounded{\includegraphics[keepaspectratio]{images/zotero1.png}}

To insert a citation:

\begin{itemize}
\item
  switch to the ``Visual Editor'' mode,
\item
  add \texttt{bibliography:\ references.bib} in the YAML header,
\item
  press the ``Insert'' button in the toolbar and select ``Citation''
\end{itemize}

A reference list or bibliography is automatically generated at the end
of the document with all the citations used.

\begin{center}\rule{0.5\linewidth}{0.5pt}\end{center}

\subsection{Adding Citations with
Zotero}\label{adding-citations-with-zotero-2}

\pandocbounded{\includegraphics[keepaspectratio]{images/zotero2.png}}

\begin{itemize}
\item
  You can select citations straight from your library and folders in
  Zotero
\item
  Each citation is assigned a short citation key based on the author and
  year (e.g., pauwels2015).
\item
  You can quickly cite it in your document by typing @ followed by the
  citation key.
\end{itemize}

\begin{center}\rule{0.5\linewidth}{0.5pt}\end{center}

\subsection{Adding Citations with
Zotero}\label{adding-citations-with-zotero-3}

Let's practice adding a citation

\textbf{Task 9}

\begin{itemize}
\tightlist
\item[$\square$]
  add \texttt{bibliography:\ references.bib} to your YAML header
\item[$\square$]
  \href{https://journals.plos.org/plosone/article?id=10.1371/journal.pone.0127819}{follow
  this link to an article on diets and broiler chicks}
\item[$\square$]
  add the citation to your Zotero Library
\end{itemize}

\begin{Shaded}
\begin{Highlighting}[]
\NormalTok{Diet affects chick body weight. In this study, low{-}energy feed made fast{-}growing chicks lighter, while slower{-}growing chicks compensated by eating more }
\end{Highlighting}
\end{Shaded}

\begin{itemize}
\tightlist
\item[$\square$]
  copy the text above and place the cursor at the end of the sentence
\end{itemize}

\emph{Instructor Note} - There are several ways to add a citation to
your Zotero library but an easy way is using the `magic wand' (adding
the item by its identifier). Here, you can just copy the DOI of the
study, hit Enter, and it will be added to Zotero. The DOIs of the
articles in these tasks will be clearly visible so this method is
recommended.

\begin{center}\rule{0.5\linewidth}{0.5pt}\end{center}

\subsection{Adding Citations with
Zotero}\label{adding-citations-with-zotero-4}

\textbf{Task 9}

\begin{itemize}
\tightlist
\item[$\square$]
  in \textbf{Visual Mode}, click ``Insert'' and find the article in the
  Zotero Library
\item[$\square$]
  click ``Insert'' to add the citation
\end{itemize}

\begin{figure}[H]

{\centering \pandocbounded{\includegraphics[keepaspectratio]{images/zotero3.pdf}}

}

\caption{When you find the article in the Zotero Library folder here,
select it and then click Insert}

\end{figure}%

\emph{Instructor Note} - It is possible to add citations in Source mode,
but it is easier and more efficient to do it in Visual mdoe. This is
also a good opportunity for students to see all the work they've done so
far in the Quarto document now in Visual mode.

\begin{center}\rule{0.5\linewidth}{0.5pt}\end{center}

\subsection{Adding Citations with
Zotero}\label{adding-citations-with-zotero-5}

Let's add another citation to practice adding \textbf{in-text} citations

\textbf{Task 10}

\begin{Shaded}
\begin{Highlighting}[]
\NormalTok{ found that broiler chicks fed a higher nutrient{-}density diet gained more weight and did so more efficiently.}
\end{Highlighting}
\end{Shaded}

\begin{itemize}
\tightlist
\item[$\square$]
  copy the text above and place the cursor at the beginning of the
  sentence
\item[$\square$]
  \href{https://journals.plos.org/plosone/article?id=10.1371/journal.pone.0153859\#:~:text=In\%20conclusion\%2C\%20HDND\%20improved\%20the,diet\%20adversely\%20affected\%20bone\%20mineralization.}{follow
  this link to another article on diets and broiler chicks}
\item[$\square$]
  add the citation to your Zotero Library
\end{itemize}

\begin{center}\rule{0.5\linewidth}{0.5pt}\end{center}

\subsection{Adding Citations with
Zotero}\label{adding-citations-with-zotero-6}

\textbf{Task 10}

\begin{itemize}
\tightlist
\item[$\square$]
  click ``Insert'' and find the article in the Zotero Library
\item[$\square$]
  check the box next to ``In-text''
\end{itemize}

\pandocbounded{\includegraphics[keepaspectratio]{images/zotero4.pdf}}

\begin{itemize}
\tightlist
\item[$\square$]
  click ``Insert'' to add the citation
\end{itemize}

\begin{center}\rule{0.5\linewidth}{0.5pt}\end{center}

\subsection{Adding Citations with
Zotero}\label{adding-citations-with-zotero-7}

Now you should have

\begin{itemize}
\tightlist
\item
  a citation,
\item
  an in-text citation, and
\item
  a reference list (bibliography).
\end{itemize}

\pandocbounded{\includegraphics[keepaspectratio]{images/zotero5.pdf}}

\begin{tcolorbox}[enhanced jigsaw, colback=white, titlerule=0mm, toprule=.15mm, breakable, colframe=quarto-callout-note-color-frame, toptitle=1mm, opacityback=0, title=\textcolor{quarto-callout-note-color}{\faInfo}\hspace{0.5em}{changing the citation style}, coltitle=black, colbacktitle=quarto-callout-note-color!10!white, leftrule=.75mm, opacitybacktitle=0.6, bottomrule=.15mm, bottomtitle=1mm, arc=.35mm, rightrule=.15mm, left=2mm]

You can change the citation style by using csl files directly from the
Zotero website. For e.g., putting
``\texttt{csl:\ https://www.zotero.org/styles/apa}'' in your YAML
changes the citation style to APA.

\end{tcolorbox}

\begin{center}\rule{0.5\linewidth}{0.5pt}\end{center}

\subsection{Publishing}\label{publishing}

\textbf{What is Publishing with Quarto?} It's the process of sharing the
rendered documents or projects online so it can become accessible to
others.

You can publish to:

\begin{itemize}
\tightlist
\item
  GitHub Pages
\item
  Quarto Pub
\item
  personal website
\end{itemize}

\begin{tcolorbox}[enhanced jigsaw, colback=white, titlerule=0mm, toprule=.15mm, breakable, colframe=quarto-callout-note-color-frame, toptitle=1mm, opacityback=0, title=\textcolor{quarto-callout-note-color}{\faInfo}\hspace{0.5em}{list of publishing services}, coltitle=black, colbacktitle=quarto-callout-note-color!10!white, leftrule=.75mm, opacitybacktitle=0.6, bottomrule=.15mm, bottomtitle=1mm, arc=.35mm, rightrule=.15mm, left=2mm]

Here is a link for a list of publishing services and more information!

\end{tcolorbox}

\emph{Speaker Note} - Publishing in Quarto is all about making your work
accessible to others. Once you've created and rendered your document,
you can share it online so that collaborators, students, or the wider
community can view it. Quarto supports multiple publishing options,
including GitHub Pages, Quarto Pub, or even your own personal website.

\emph{Instructor Note} - The key takeaway here is that publishing turns
your local work into something that can be accessed from anywhere.

\begin{center}\rule{0.5\linewidth}{0.5pt}\end{center}

\subsection{Publishing to GitHub}\label{publishing-to-github}

A great option to share your document is publishing to \textbf{GitHub}.

\begin{itemize}
\item
  \textbf{GitHub} is a platform for hosting and sharing code and
  projects online
\item
  An advantage of GitHub is it allows for \textbf{version control}- the
  ability to track and manage changes over time
\item
  A free and widely used way to publish your Quarto document to GitHub
  is by using \textbf{GitHub Pages}
\end{itemize}

\begin{center}\rule{0.5\linewidth}{0.5pt}\end{center}

\subsection{Publishing to GitHub}\label{publishing-to-github-1}

\textbf{GitHub Pages} enables you to publish content based on source
code managed within a GitHub repository.

\textbf{3 methods} to publish a Quarto document to GitHub Pages:

\begin{enumerate}
\def\labelenumi{\arabic{enumi}.}
\tightlist
\item
  Render to the \texttt{docs} directory and checking it into your
  repository
\item
  Use the \texttt{quarto\ publish} command
\item
  Use GitHub Actions to auto-render and publish whenever you push
  changes
\end{enumerate}

\begin{center}\rule{0.5\linewidth}{0.5pt}\end{center}

\subsection{Publishing to GitHub}\label{publishing-to-github-2}

Let's publish our document using the first method:

\textbf{Render to \texttt{docs}}

\begin{center}\rule{0.5\linewidth}{0.5pt}\end{center}

\subsection{Publishing to GitHub}\label{publishing-to-github-3}

Work with \textbf{Git Bash} directly from the \textbf{Terminal} in
RStudio.

\begin{itemize}
\tightlist
\item
  click Terminal → Terminal Options and set it to open with Git Bash
\end{itemize}

\pandocbounded{\includegraphics[keepaspectratio]{images/gitpages1.pdf}}

\pandocbounded{\includegraphics[keepaspectratio]{images/gitpages2.pdf}}

\begin{itemize}
\tightlist
\item
  then, click Terminal → New Terminal to begin working with Git Bash
\end{itemize}

\emph{Instructor Note} - The last step (to open a new Terminal) is
important because the current Terminal does not just automatically
change to Git Bash.

\begin{center}\rule{0.5\linewidth}{0.5pt}\end{center}

\subsection{Publishing to GitHub}\label{publishing-to-github-4}

\textbf{Task 11}

\begin{itemize}
\tightlist
\item[$\square$]
  in Quarto Document, edit YAML and add:
\end{itemize}

\begin{Shaded}
\begin{Highlighting}[]
\NormalTok{project:}
\NormalTok{  type: website}
\NormalTok{  output{-}dir: docs}
\end{Highlighting}
\end{Shaded}

\begin{itemize}
\tightlist
\item[$\square$]
  set Terminal to open with Git Bash and open new Terminal
\item[$\square$]
  change directory to the path of the main folder with the Quarto files
  using \texttt{cd}
\end{itemize}

\emph{Instructor Note} - Reminder that copying \& pasting with keyboard
shortcuts do not work the same in Git Bash. Right-clicking and selecting
Paste is an easy option. - Setting the directory can be tricky. -
Backslashes (\texttt{\textbackslash{}}) need to be changes to forward
slashes (\texttt{/}). - If you're pasting the exact path of the folder,
wrap it in quotation marks (\texttt{"..."})

\begin{center}\rule{0.5\linewidth}{0.5pt}\end{center}

\subsection{Publishing to GitHub}\label{publishing-to-github-5}

\textbf{Task 11}

\begin{Shaded}
\begin{Highlighting}[]
\NormalTok{echo "project:}
\NormalTok{  type: website}
\NormalTok{  output{-}dir: docs" \textgreater{} \_quarto.yml}
\end{Highlighting}
\end{Shaded}

\begin{itemize}
\tightlist
\item[$\square$]
  make a simple \texttt{\_quarto.yml} file by copying \& pasting the
  above in the Terminal
\end{itemize}

\begin{Shaded}
\begin{Highlighting}[]
\NormalTok{touch .nojekyll}
\end{Highlighting}
\end{Shaded}

\begin{itemize}
\tightlist
\item[$\square$]
  copy \& paste the above into the Terminal to add a \texttt{.nojekyll}
  file
\end{itemize}

\begin{tcolorbox}[enhanced jigsaw, colback=white, titlerule=0mm, toprule=.15mm, breakable, colframe=quarto-callout-note-color-frame, toptitle=1mm, opacityback=0, title=\textcolor{quarto-callout-note-color}{\faInfo}\hspace{0.5em}{.nojekyll}, coltitle=black, colbacktitle=quarto-callout-note-color!10!white, leftrule=.75mm, opacitybacktitle=0.6, bottomrule=.15mm, bottomtitle=1mm, arc=.35mm, rightrule=.15mm, left=2mm]

Adding a \texttt{.nojekyll} file to the root of your repo tells GitHub
Pages not to do additional processing of your published site using
Jekyll

\end{tcolorbox}

\begin{center}\rule{0.5\linewidth}{0.5pt}\end{center}

\subsection{Publishing to GitHub}\label{publishing-to-github-6}

\textbf{Task 11}

\begin{Shaded}
\begin{Highlighting}[]
\NormalTok{quarto render}
\end{Highlighting}
\end{Shaded}

\begin{itemize}
\tightlist
\item[$\square$]
  copy \& paste the above into Terminal to generate HTML in the
  \texttt{docs/} folder
\end{itemize}

\begin{Shaded}
\begin{Highlighting}[]
\NormalTok{git init}
\NormalTok{git branch {-}M main  }
\end{Highlighting}
\end{Shaded}

\begin{itemize}
\tightlist
\item[$\square$]
  copy \& paste the above into Terminal to initialize Git and rename
  ``master'' to ``main''
\end{itemize}

\emph{Instructor Note} - You will see a warning about
\texttt{references.bib\ not\ found\ in\ resource\ path} which usually
happens because Quarto looks for the file in the project's root or
resource path, but Git Bash is case-sensitive and strict with paths. -
So, you will see in your completed Git Pages web page that the citations
look strange but we will leave it like that for now. - Extra steps are
needed to correct this that will be covered in the Advanced Intro to
Quarto Activity PDF.

\begin{center}\rule{0.5\linewidth}{0.5pt}\end{center}

\subsection{Publishing to GitHub}\label{publishing-to-github-7}

\textbf{Task 11}

\begin{itemize}
\tightlist
\item[$\square$]
  in \href{github.com}{GitHub}, create a new repo
\item[$\square$]
  in Terminal, add remote by using ``\texttt{git\ remote\ add\ origin}''
  + the repo's HTTPS URL (e.g.,
  ``\texttt{git\ remote\ add\ origin\ https://github.com/yourname/chickweight-exercise.git}'')
\end{itemize}

\begin{Shaded}
\begin{Highlighting}[]
\NormalTok{quarto render}
\NormalTok{git add docs}
\NormalTok{git commit {-}m "Publish site to docs/"}
\NormalTok{git push {-}u origin main}
\end{Highlighting}
\end{Shaded}

\begin{itemize}
\tightlist
\item[$\square$]
  copy \& paste the above into Terminal to render your site and push it
  to GitHub
\end{itemize}

\emph{Instructor Note} - For this to work, the repo created must be set
to Public

\begin{center}\rule{0.5\linewidth}{0.5pt}\end{center}

\subsection{Publishing to GitHub}\label{publishing-to-github-8}

\begin{itemize}
\tightlist
\item[$\square$]
  In your repo on \href{github.com}{GitHub}, go to Settings
\item[$\square$]
  Click ``Pages'' to configure GitHub Pages
\end{itemize}

\pandocbounded{\includegraphics[keepaspectratio]{images/gitpages3.png}}

\begin{itemize}
\tightlist
\item[$\square$]
  change ``none'' to ``main''
\end{itemize}

\pandocbounded{\includegraphics[keepaspectratio]{images/gitpages4.png}}

\begin{itemize}
\tightlist
\item[$\square$]
  change ``/root'' to ``/docs'' and click Save
\end{itemize}

After 1-2 minutes, you will see that your site is \textbf{live}.

\emph{Instructor Note} - Reminder that the citations will not be
complete because configuring to include the bibliography was not done.
That's an additional step.

\begin{center}\rule{0.5\linewidth}{0.5pt}\end{center}

\subsection{Recap}\label{recap}

\begin{center}\rule{0.5\linewidth}{0.5pt}\end{center}

\subsection{Relevance and
Implications}\label{relevance-and-implications}

Let's discuss how useful \textbf{Quarto} can be for \textbf{open
research}.

Consider the following questions:

\begin{itemize}
\item
  How can using Quarto improve transparency and reproducibility in your
  work?
\item
  In what ways does Quarto make sharing code and data with collaborators
  easier?
\item
  Do you see advantages in using Quarto compared to traditional tools
  like Word or PowerPoint?
\item
  What challenges might you face in adopting Quarto in your current
\end{itemize}

\emph{Instructor Note} - Aim: To work out the relevance of the topic to
your students. - In an interactive setting, discuss how the new skills
could be applied in practise with specific examples. - Examine downfalls
and practical obstacles.

\begin{center}\rule{0.5\linewidth}{0.5pt}\end{center}

\subsection{What is the take-home
message?}\label{what-is-the-take-home-message}

Let's wrap up by identifying the \textbf{key takeaway} from today's
session together.

Ask yourself: \textbf{``If you had to explain to a colleague in one
sentence why learning Quarto matters for research and teaching, what
would you say?''}

\emph{Instructor Note} - Aim: End lesson on clear take-home message that
are interactively compiled by students. - Encourage students to share
one sentence take-home messages. You can collect them verbally or in a
shared doc/board. - Reinforce that Quarto is not just a tool, but a
mindset shift toward openness and collaboration.

\begin{center}\rule{0.5\linewidth}{0.5pt}\end{center}

\subsection{Assignment}\label{assignment}

\begin{itemize}
\tightlist
\item
  \textbf{Aim}: Explain the homework assignment and the rationale behind
  the homework.
\item
  Examine whether/how it will be assessed
\item
  Mention scoring rubrics, if applicable
\item
  Design a peer-review system for assignments to place students in role
  of reviewer and author
\end{itemize}

\begin{center}\rule{0.5\linewidth}{0.5pt}\end{center}

\subsection{To conclude: Survey time!}\label{to-conclude-survey-time}

\emph{Instructor Note}

\begin{itemize}
\tightlist
\item
  Aim: This post-submodule survey serves to examine students' current
  knowledge about the sumodule's topic.
\item
  Use free survey software such as or other survey software (particify,
  formR) to establish the following questions (these are examples that
  can be adapted to suit your needs):
\end{itemize}

\textbf{What is your level of familiarity with Quarto before this
lesson?} a) I've never used it b) I've heard of it but never tried it c)
I've experimented a little d) I use it occasionally e) I use it
frequently

\textbf{Which of the following best describes Quarto?} a) A programming
language b) A publishing system for reproducible documents ✅ c) A code
editor d) A type of version control system

\textbf{Which components are combined in the authoring process of Quarto
documents?} a) Only YAML b) Only Markdown c) Neither d) YAML and
Markdown together ✅

\textbf{Which of the following modes can be used to create a Quarto
document?} a) Source mode b) Visual mode c) Both ✅ d) Neither

\textbf{On a scale of 1 to 5, how comfortable are you switching between
Source and Visual modes?}

\textbf{Which of these are examples of Markdown text formatting? (Select
all that apply)} a) Bold text ✅ b) Italic text ✅ c) Adding code chunks
d) Adding hyperlinks ✅

\textbf{Why should you use text formatting in Quarto documents?} a) For
aesthetics only b) To improve clarity and emphasis ✅ c) To make the
file larger d) To change the coding language

\textbf{When would you consider using multi-column layouts?} a) To
shorten the document b) To make information more visually organized ✅
c) To add footnotes only d) To increase file size

\textbf{What is the purpose of code chunks in Quarto documents?} a) To
include executable code and results within the document ✅ b) To store
references c) To format text d) To hide metadata

\textbf{Which YAML options relate to how code chunks are displayed?} a)
echo: true ✅ b) code-fold: false ✅ c) title: ``My Document'' d)
format: html

\textbf{Why is including code in research documents important?} a) To
make the document more difficult to read b) To ensure only the author
can reproduce results c) To replace data collection d) To make research
reproducible and transparent ✅

\textbf{Which of the following are common publishing options for Quarto
projects? (Select all that apply)} a) GitHub Pages ✅ b) Quarto Pub ✅
c) Social media platforms d) Personal websites ✅

\textbf{What is the benefit of publishing your Quarto documents?} a)
They make your computer run faster b) They allow offline editing only c)
They become accessible to others for sharing and collaboration ✅ d)
They automatically write the code for you

\begin{center}\rule{0.5\linewidth}{0.5pt}\end{center}

\subsection{Discussion of survey
results}\label{discussion-of-survey-results-1}

\emph{Instructor Note}

\begin{itemize}
\tightlist
\item
  Aim: Briefly examine the answers given to each question interactively
  with the group.
\item
  Compare and highlight specific differences in answers between pre- and
  post-survey answers
\end{itemize}

\begin{center}\rule{0.5\linewidth}{0.5pt}\end{center}

\subsection{References}\label{references}

\begin{itemize}
\tightlist
\item
  Provide literature you refer to throughout this lesson.
\end{itemize}

\begin{center}\rule{0.5\linewidth}{0.5pt}\end{center}

\section{\texorpdfstring{Thanks! }{Thanks!  }}\label{thanks}

See you next class :)

\begin{center}\rule{0.5\linewidth}{0.5pt}\end{center}

\subsection{Pedagogical add-on tools for
instructors}\label{pedagogical-add-on-tools-for-instructors}

\begin{itemize}
\tightlist
\item
  This section is dedicated to ideas on how to incorporate pedagogical
  tools into teaching for this specific submodule topic. This could
  mean:

  \begin{itemize}
  \tightlist
  \item
    Information about the scientific evidence on the theory of the
    pedagogical add-on tool and the evidence for its efficacy.
  \item
    Discussion/reflection on how tools can be incorporated into the
    teaching for this particular content.
  \item
    Extra exercises for faster students.
  \end{itemize}
\end{itemize}

\begin{center}\rule{0.5\linewidth}{0.5pt}\end{center}

\subsection{Additional literature for
instructors}\label{additional-literature-for-instructors}

\begin{itemize}
\tightlist
\item
  References for content
\item
  References for pedagogical add-on tools\\
\item
  Other resources (videos etc.)
\end{itemize}

\begin{center}\rule{0.5\linewidth}{0.5pt}\end{center}

\subsection{Attribution and license
details}\label{attribution-and-license-details}

\begin{itemize}
\item
  This slide should contain information about the license and
  attribution details of this current set of slides.
\item
  The default for the created materials is
  \href{https://creativecommons.org/licenses/by/4.0/}{CC-BY-SA 4.0}

  \begin{itemize}
  \tightlist
  \item
    = Creative Commons license that allows others to \textbf{share,
    adapt, and build upon} the original work
  \item
    \textbf{only} if they attribute the creator and also share their new
    work under the \textbf{same terms}
  \item
    allows for both \textbf{commercial and non-commercial} use of the
    licensed material
  \end{itemize}
\item
  Components of attributions:

  \begin{itemize}
  \tightlist
  \item
    \texttt{Title}
  \item
    \texttt{Author}
  \item
    \texttt{Source}
  \item
    \texttt{License}
  \end{itemize}
\end{itemize}

\begin{center}\rule{0.5\linewidth}{0.5pt}\end{center}

\subsection{Example attribution (for previous
slide)}\label{example-attribution-for-previous-slide}

``\href{https://github.com/SvonGrebmer/Template-Slides-Students}{Tutorial
template for student track}'' by Sarah von Grebmer is licensed under
\href{https://creativecommons.org/licenses/by/4.0/}{CC-BY-SA 4.0}.

\begin{center}\rule{0.5\linewidth}{0.5pt}\end{center}




\end{document}
