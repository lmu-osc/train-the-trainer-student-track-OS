% Options for packages loaded elsewhere
% Options for packages loaded elsewhere
\PassOptionsToPackage{unicode}{hyperref}
\PassOptionsToPackage{hyphens}{url}
\PassOptionsToPackage{dvipsnames,svgnames,x11names}{xcolor}
%
\documentclass[
  letterpaper,
  DIV=11,
  numbers=noendperiod]{scrartcl}
\usepackage{xcolor}
\usepackage[top=30mm,left=30mm]{geometry}
\usepackage{amsmath,amssymb}
\setcounter{secnumdepth}{-\maxdimen} % remove section numbering
\usepackage{iftex}
\ifPDFTeX
  \usepackage[T1]{fontenc}
  \usepackage[utf8]{inputenc}
  \usepackage{textcomp} % provide euro and other symbols
\else % if luatex or xetex
  \usepackage{unicode-math} % this also loads fontspec
  \defaultfontfeatures{Scale=MatchLowercase}
  \defaultfontfeatures[\rmfamily]{Ligatures=TeX,Scale=1}
\fi
\usepackage{lmodern}
\ifPDFTeX\else
  % xetex/luatex font selection
\fi
% Use upquote if available, for straight quotes in verbatim environments
\IfFileExists{upquote.sty}{\usepackage{upquote}}{}
\IfFileExists{microtype.sty}{% use microtype if available
  \usepackage[]{microtype}
  \UseMicrotypeSet[protrusion]{basicmath} % disable protrusion for tt fonts
}{}
\makeatletter
\@ifundefined{KOMAClassName}{% if non-KOMA class
  \IfFileExists{parskip.sty}{%
    \usepackage{parskip}
  }{% else
    \setlength{\parindent}{0pt}
    \setlength{\parskip}{6pt plus 2pt minus 1pt}}
}{% if KOMA class
  \KOMAoptions{parskip=half}}
\makeatother
% Make \paragraph and \subparagraph free-standing
\makeatletter
\ifx\paragraph\undefined\else
  \let\oldparagraph\paragraph
  \renewcommand{\paragraph}{
    \@ifstar
      \xxxParagraphStar
      \xxxParagraphNoStar
  }
  \newcommand{\xxxParagraphStar}[1]{\oldparagraph*{#1}\mbox{}}
  \newcommand{\xxxParagraphNoStar}[1]{\oldparagraph{#1}\mbox{}}
\fi
\ifx\subparagraph\undefined\else
  \let\oldsubparagraph\subparagraph
  \renewcommand{\subparagraph}{
    \@ifstar
      \xxxSubParagraphStar
      \xxxSubParagraphNoStar
  }
  \newcommand{\xxxSubParagraphStar}[1]{\oldsubparagraph*{#1}\mbox{}}
  \newcommand{\xxxSubParagraphNoStar}[1]{\oldsubparagraph{#1}\mbox{}}
\fi
\makeatother

\usepackage{color}
\usepackage{fancyvrb}
\newcommand{\VerbBar}{|}
\newcommand{\VERB}{\Verb[commandchars=\\\{\}]}
\DefineVerbatimEnvironment{Highlighting}{Verbatim}{commandchars=\\\{\}}
% Add ',fontsize=\small' for more characters per line
\usepackage{framed}
\definecolor{shadecolor}{RGB}{241,243,245}
\newenvironment{Shaded}{\begin{snugshade}}{\end{snugshade}}
\newcommand{\AlertTok}[1]{\textcolor[rgb]{0.68,0.00,0.00}{#1}}
\newcommand{\AnnotationTok}[1]{\textcolor[rgb]{0.37,0.37,0.37}{#1}}
\newcommand{\AttributeTok}[1]{\textcolor[rgb]{0.40,0.45,0.13}{#1}}
\newcommand{\BaseNTok}[1]{\textcolor[rgb]{0.68,0.00,0.00}{#1}}
\newcommand{\BuiltInTok}[1]{\textcolor[rgb]{0.00,0.23,0.31}{#1}}
\newcommand{\CharTok}[1]{\textcolor[rgb]{0.13,0.47,0.30}{#1}}
\newcommand{\CommentTok}[1]{\textcolor[rgb]{0.37,0.37,0.37}{#1}}
\newcommand{\CommentVarTok}[1]{\textcolor[rgb]{0.37,0.37,0.37}{\textit{#1}}}
\newcommand{\ConstantTok}[1]{\textcolor[rgb]{0.56,0.35,0.01}{#1}}
\newcommand{\ControlFlowTok}[1]{\textcolor[rgb]{0.00,0.23,0.31}{\textbf{#1}}}
\newcommand{\DataTypeTok}[1]{\textcolor[rgb]{0.68,0.00,0.00}{#1}}
\newcommand{\DecValTok}[1]{\textcolor[rgb]{0.68,0.00,0.00}{#1}}
\newcommand{\DocumentationTok}[1]{\textcolor[rgb]{0.37,0.37,0.37}{\textit{#1}}}
\newcommand{\ErrorTok}[1]{\textcolor[rgb]{0.68,0.00,0.00}{#1}}
\newcommand{\ExtensionTok}[1]{\textcolor[rgb]{0.00,0.23,0.31}{#1}}
\newcommand{\FloatTok}[1]{\textcolor[rgb]{0.68,0.00,0.00}{#1}}
\newcommand{\FunctionTok}[1]{\textcolor[rgb]{0.28,0.35,0.67}{#1}}
\newcommand{\ImportTok}[1]{\textcolor[rgb]{0.00,0.46,0.62}{#1}}
\newcommand{\InformationTok}[1]{\textcolor[rgb]{0.37,0.37,0.37}{#1}}
\newcommand{\KeywordTok}[1]{\textcolor[rgb]{0.00,0.23,0.31}{\textbf{#1}}}
\newcommand{\NormalTok}[1]{\textcolor[rgb]{0.00,0.23,0.31}{#1}}
\newcommand{\OperatorTok}[1]{\textcolor[rgb]{0.37,0.37,0.37}{#1}}
\newcommand{\OtherTok}[1]{\textcolor[rgb]{0.00,0.23,0.31}{#1}}
\newcommand{\PreprocessorTok}[1]{\textcolor[rgb]{0.68,0.00,0.00}{#1}}
\newcommand{\RegionMarkerTok}[1]{\textcolor[rgb]{0.00,0.23,0.31}{#1}}
\newcommand{\SpecialCharTok}[1]{\textcolor[rgb]{0.37,0.37,0.37}{#1}}
\newcommand{\SpecialStringTok}[1]{\textcolor[rgb]{0.13,0.47,0.30}{#1}}
\newcommand{\StringTok}[1]{\textcolor[rgb]{0.13,0.47,0.30}{#1}}
\newcommand{\VariableTok}[1]{\textcolor[rgb]{0.07,0.07,0.07}{#1}}
\newcommand{\VerbatimStringTok}[1]{\textcolor[rgb]{0.13,0.47,0.30}{#1}}
\newcommand{\WarningTok}[1]{\textcolor[rgb]{0.37,0.37,0.37}{\textit{#1}}}

\usepackage{longtable,booktabs,array}
\usepackage{calc} % for calculating minipage widths
% Correct order of tables after \paragraph or \subparagraph
\usepackage{etoolbox}
\makeatletter
\patchcmd\longtable{\par}{\if@noskipsec\mbox{}\fi\par}{}{}
\makeatother
% Allow footnotes in longtable head/foot
\IfFileExists{footnotehyper.sty}{\usepackage{footnotehyper}}{\usepackage{footnote}}
\makesavenoteenv{longtable}
\usepackage{graphicx}
\makeatletter
\newsavebox\pandoc@box
\newcommand*\pandocbounded[1]{% scales image to fit in text height/width
  \sbox\pandoc@box{#1}%
  \Gscale@div\@tempa{\textheight}{\dimexpr\ht\pandoc@box+\dp\pandoc@box\relax}%
  \Gscale@div\@tempb{\linewidth}{\wd\pandoc@box}%
  \ifdim\@tempb\p@<\@tempa\p@\let\@tempa\@tempb\fi% select the smaller of both
  \ifdim\@tempa\p@<\p@\scalebox{\@tempa}{\usebox\pandoc@box}%
  \else\usebox{\pandoc@box}%
  \fi%
}
% Set default figure placement to htbp
\def\fps@figure{htbp}
\makeatother

\ifLuaTeX
  \usepackage{luacolor}
  \usepackage[soul]{lua-ul}
\else
  \usepackage{soul}
\fi




\setlength{\emergencystretch}{3em} % prevent overfull lines

\providecommand{\tightlist}{%
  \setlength{\itemsep}{0pt}\setlength{\parskip}{0pt}}



 


\KOMAoption{captions}{tableheading}
\makeatletter
\@ifpackageloaded{tcolorbox}{}{\usepackage[skins,breakable]{tcolorbox}}
\@ifpackageloaded{fontawesome5}{}{\usepackage{fontawesome5}}
\definecolor{quarto-callout-color}{HTML}{909090}
\definecolor{quarto-callout-note-color}{HTML}{0758E5}
\definecolor{quarto-callout-important-color}{HTML}{CC1914}
\definecolor{quarto-callout-warning-color}{HTML}{EB9113}
\definecolor{quarto-callout-tip-color}{HTML}{00A047}
\definecolor{quarto-callout-caution-color}{HTML}{FC5300}
\definecolor{quarto-callout-color-frame}{HTML}{acacac}
\definecolor{quarto-callout-note-color-frame}{HTML}{4582ec}
\definecolor{quarto-callout-important-color-frame}{HTML}{d9534f}
\definecolor{quarto-callout-warning-color-frame}{HTML}{f0ad4e}
\definecolor{quarto-callout-tip-color-frame}{HTML}{02b875}
\definecolor{quarto-callout-caution-color-frame}{HTML}{fd7e14}
\makeatother
\makeatletter
\@ifpackageloaded{caption}{}{\usepackage{caption}}
\AtBeginDocument{%
\ifdefined\contentsname
  \renewcommand*\contentsname{Table of contents}
\else
  \newcommand\contentsname{Table of contents}
\fi
\ifdefined\listfigurename
  \renewcommand*\listfigurename{List of Figures}
\else
  \newcommand\listfigurename{List of Figures}
\fi
\ifdefined\listtablename
  \renewcommand*\listtablename{List of Tables}
\else
  \newcommand\listtablename{List of Tables}
\fi
\ifdefined\figurename
  \renewcommand*\figurename{Figure}
\else
  \newcommand\figurename{Figure}
\fi
\ifdefined\tablename
  \renewcommand*\tablename{Table}
\else
  \newcommand\tablename{Table}
\fi
}
\@ifpackageloaded{float}{}{\usepackage{float}}
\floatstyle{ruled}
\@ifundefined{c@chapter}{\newfloat{codelisting}{h}{lop}}{\newfloat{codelisting}{h}{lop}[chapter]}
\floatname{codelisting}{Listing}
\newcommand*\listoflistings{\listof{codelisting}{List of Listings}}
\makeatother
\makeatletter
\makeatother
\makeatletter
\@ifpackageloaded{caption}{}{\usepackage{caption}}
\@ifpackageloaded{subcaption}{}{\usepackage{subcaption}}
\makeatother
\usepackage{bookmark}
\IfFileExists{xurl.sty}{\usepackage{xurl}}{} % add URL line breaks if available
\urlstyle{same}
\hypersetup{
  pdftitle={Master Template Slides: Title of submodule},
  pdfauthor={LMU Open Science Center},
  colorlinks=true,
  linkcolor={blue},
  filecolor={Maroon},
  citecolor={Blue},
  urlcolor={Blue},
  pdfcreator={LaTeX via pandoc}}


\title{Master Template Slides: Title of submodule}
\author{LMU Open Science Center}
\date{06/01/2026}
\begin{document}
\maketitle


\subsection{Contribution statement and
licence}\label{contribution-statement-and-licence}

\textbf{Creator}: Name, first name
(\pandocbounded{\includegraphics[keepaspectratio]{OS-template-slides_files/mediabag/orcid_16x16.png}}
orcid)

\textbf{Reviewer}: Name, first name
(\pandocbounded{\includegraphics[keepaspectratio]{OS-template-slides_files/mediabag/orcid_16x16.png}}
orcid)

\textbf{Acknowledgments}: Name, first name
(\pandocbounded{\includegraphics[keepaspectratio]{OS-template-slides_files/mediabag/orcid_16x16.png}}
orcid)

This work by XXXX XXXX, Sara Lil Middleton and Sarah von Grebmer zu
Wolfsthurn is licensed under a CC-BY 4.0
\href{https://creativecommons.org/licenses/by/4.0/deed.en}{Creative
Commons Attribution 4.0 International License}.

\textbf{Speaker Notes}: These are the \textbf{speaker notes}. You will a
script for the presenter for every slide. In presentation mode, your
audience will not be able to see these speaker notes, they are only
visible to the presenter.

\textbf{Instructor Notes}: There are also \textbf{instructor notes}. For
some slides, there will be pedagogical tips, suggestons for acitivities
and troubleshooting tips for issues your audience might run into. You
can find these notes underneath the speaker notes.

\textbf{Acessibility Tips}: Where applicable, this is a space to add any
tips you may have to facilitate the accessibility of your slides and
activities.

\begin{center}\rule{0.5\linewidth}{0.5pt}\end{center}

\subsection{Prerequisites}\label{prerequisites}

\begin{tcolorbox}[enhanced jigsaw, toptitle=1mm, bottomtitle=1mm, leftrule=.75mm, bottomrule=.15mm, colframe=quarto-callout-important-color-frame, colback=white, arc=.35mm, opacitybacktitle=0.6, toprule=.15mm, coltitle=black, colbacktitle=quarto-callout-important-color!10!white, breakable, left=2mm, title=\textcolor{quarto-callout-important-color}{\faExclamation}\hspace{0.5em}{Important}, opacityback=0, titlerule=0mm, rightrule=.15mm]

Before completing this submodule, please carefully read about the
prerequisites.

\end{tcolorbox}

\begin{longtable}[]{@{}
  >{\raggedright\arraybackslash}p{(\linewidth - 4\tabcolsep) * \real{0.3333}}
  >{\raggedright\arraybackslash}p{(\linewidth - 4\tabcolsep) * \real{0.3333}}
  >{\raggedright\arraybackslash}p{(\linewidth - 4\tabcolsep) * \real{0.3333}}@{}}
\toprule\noalign{}
\begin{minipage}[b]{\linewidth}\raggedright
Prerequisite
\end{minipage} & \begin{minipage}[b]{\linewidth}\raggedright
Description
\end{minipage} & \begin{minipage}[b]{\linewidth}\raggedright
Link/Where to find it
\end{minipage} \\
\midrule\noalign{}
\endhead
\bottomrule\noalign{}
\endlastfoot
Topic Name & Basic intro to X & Module + Submodule \\
Software Name & Configuring the environment &
\href{https://quarto.org}{Download Link} \\
\end{longtable}

\textbf{Speaker Notes}: Script for the slide here. \textbf{Instructor
Notes}: These are the prerequisites for this submodule. Before you get
started on this submodule with your audience, you need to ensure that
the audience fulfills these criteria. Outline any essential
prerequisites (software, tools, other submodules etc.) here in table
format. If you prefer bullet points to list the prerequisites, delete
the table and use bullet points instead.

\begin{center}\rule{0.5\linewidth}{0.5pt}\end{center}

\subsection{Questions from previous
submodule}\label{questions-from-previous-submodule}

\begin{itemize}
\tightlist
\item
  \textbf{Aim}: This first slide is dedicated to clarifying questions
  from the previous submodule and/or to discuss assignments.
\item
  Additional slides may need to be added depending on the nature of the
  homework assignments.
\item
  Critical for the learning process to ensure that students are on the
  same page and have been able to achieve the learning goals of the
  previous workshop.
\item
  Not applicable if this set of slides corresponds to the first
  submodule of a new module.
\end{itemize}

\textbf{Speaker Notes}: Script for the slide here. \textbf{Instructor
Notes}: - \textbf{Aim}: This first slide is dedicated to clarifying
questions from the previous submodule and/or to discuss assignments. -
Additional slides may need to be added depending on the nature of the
homework assignments. - Critical for the learning process to ensure that
students are on the same page and have been able to achieve the learning
goals of the previous workshop. - Not applicable if this set of slides
corresponds to the first submodule of a new module.

\begin{center}\rule{0.5\linewidth}{0.5pt}\end{center}

\subsection{Before we start: Survey
time!}\label{before-we-start-survey-time}

\begin{itemize}
\tightlist
\item
  \textbf{Aim}: The pre-submodule survey serves to examine students'
  prior knowledge about the submodule's topic.
\item
  Use free survey software such as particify or formR to establish the
  following questions (shown on separate slides).
\end{itemize}

\textbf{Speaker Notes}: Script for the slide here. \textbf{Instructor
Notes}: - \textbf{Aim}: The pre-submodule survey serves to examine
students' prior knowledge about the submodule's topic. - Use free survey
software such as particify or formR to establish the following questions
(shown on separate slides).

\begin{center}\rule{0.5\linewidth}{0.5pt}\end{center}

\textbf{What is your level of familiarity with {[}Topic{]} (e.g., basic
concepts, terminology, or tools)?}

\begin{enumerate}
\def\labelenumi{\alph{enumi}.}
\item
  I have never heard of it before.
\item
  I have heard of it but have never worked with it.
\item
  I have basic understanding and experience with it.
\item
  I am very familiar and have worked with it extensively.
\end{enumerate}

\begin{center}\rule{0.5\linewidth}{0.5pt}\end{center}

\textbf{Which of the following concepts or skills do you feel most
confident about in relation to {[}Topic{]}? (Select all that apply)}

\begin{enumerate}
\def\labelenumi{\alph{enumi}.}
\item
  Concept 1
\item
  Concept 2
\item
  Concept 3
\item
  Concept 4
\item
  I am not sure about any of these concepts.
\end{enumerate}

\begin{center}\rule{0.5\linewidth}{0.5pt}\end{center}

\textbf{On a scale of 1 to 5, how comfortable are you with using
{[}specific tool/technology{]} related to {[}Topic{]}? (1 = Not
comfortable at all, 5 = Very comfortable)}

\begin{enumerate}
\def\labelenumi{\alph{enumi}.}
\item
  1
\item
  2
\item
  3
\item
  4
\item
  5
\end{enumerate}

\begin{center}\rule{0.5\linewidth}{0.5pt}\end{center}

\subsection{Discussion of survey
results}\label{discussion-of-survey-results}

\begin{itemize}
\tightlist
\item
  \textbf{Aim''}: Briefly examine the answers given to each question
  interactively with the group.
\item
  Use visuals from the survey to highlight specific answers.
\end{itemize}

\textbf{Speaker Notes}: Script for the slide here. \textbf{Instructor
Notes}: - \textbf{Aim''}: Briefly examine the answers given to each
question interactively with the group. - Use visuals from the survey to
highlight specific answers.

\begin{center}\rule{0.5\linewidth}{0.5pt}\end{center}

\subsection{Where are we at?}\label{where-are-we-at}

\textbf{Previously}:

\begin{itemize}
\tightlist
\item
  Point 1
\item
  Point 2
\end{itemize}

\textbf{Next}:

\begin{itemize}
\tightlist
\item
  Point 1
\item
  Point 2
\end{itemize}

\textbf{Speaker Notes}: Script for the slide here. \textbf{Instructor
Notes}: - \textbf{Aim}: Place the topic of the current submodule within
a broader context. - Remind students what you are working towards and
what the bigger picture is.

\begin{center}\rule{0.5\linewidth}{0.5pt}\end{center}

\subsection{Covered in this session}\label{covered-in-this-session}

\begin{itemize}
\tightlist
\item
  \textbf{Aim}: This slides serves as an overview of the topics that are
  discussed, presented as bullet point:
\item
  Topic 1
\item
  Topic 2
\item
  Topic 3
\end{itemize}

\textbf{Speaker Notes}: Script for the slide here.

\begin{center}\rule{0.5\linewidth}{0.5pt}\end{center}

\subsection{Learning goals}\label{learning-goals}

\begin{itemize}
\item
  \textbf{Aim}: Formulate specific, action-oriented goals learning goals
  which are measurable and observable in line with Bloom's taxonomy
  (Anderson et al., 2001; Bloom et al., 1956)
\item
  Place an emphasis on the \textbf{verbs} of the learning goals and
  choose verbs that align with the skills you want to develop or assess.
\item
  Examples:

  \begin{itemize}
  \tightlist
  \item
    Students will \textbf{describe} the process of photosynthesis or
  \item
    Students will \textbf{construct} a diagram illustrating the process
    of photosynthesis
  \end{itemize}
\end{itemize}

\textbf{Speaker Notes}: Script for the slide here. \textbf{Instructor
Notes}: - \textbf{Aim}: Formulate specific, action-oriented goals
learning goals which are measurable and observable in line with Bloom's
taxonomy (Anderson et al., 2001; Bloom et al., 1956) - Place an emphasis
on the \textbf{verbs} of the learning goals and choose verbs that align
with the skills you want to develop or assess. - Examples: - Students
will \textbf{describe} the process of photosynthesis or - Students will
\textbf{construct} a diagram illustrating the process of photosynthesis

\begin{center}\rule{0.5\linewidth}{0.5pt}\end{center}

\subsection{Key terms and definitions}\label{key-terms-and-definitions}

\begin{itemize}
\item
  \textbf{Aim}: Introduce key terms and definitions that students will
  come across throughout the session.
\item
  \textbf{Key Term 1}: Definition
\item
  \textbf{Key Term 2}: Definition
\item
  \textbf{Key Term 3}: Definition
\end{itemize}

\textbf{Speaker Notes}: Add script for slide here. \textbf{Instructor
Notes}: Base yourself on conceptual change theory and examine existing
concepts in relation to some key terms. Re-examine the formation of new
concepts at the end of the lesson.

\begin{center}\rule{0.5\linewidth}{0.5pt}\end{center}

\subsection{Introduction of submodule
topic}\label{introduction-of-submodule-topic}

\begin{itemize}
\tightlist
\item
  \textbf{Aim}: Core theoretical introduction of submodule topic.
\item
  Pair theoretical aspects with practical exercises and group
  discussions according to the Think-Pair-Share style and according to
  Cognitive Load Theory (Sweller, 1980).
\item
  Use multiple slides for this part.
\end{itemize}

\textbf{Speaker notes}: Add script for the slide here.
\textbf{Instructor Notes}: - \textbf{Aim}: Core theoretical introduction
of submodule topic. - For a 90-minute lesson, the instructor should try
to ``lecture'' for only 20 minutes, students should work in
groups/pairs/on their own for at least 55 minutes of the lesson (+ a 15
minute break). - Pair theoretical aspects with practical exercises and
group discussions according to the Think-Pair-Share style and according
to Cognitive Load Theory (Sweller, 1980). - Use multiple slides for this
part.

\begin{center}\rule{0.5\linewidth}{0.5pt}\end{center}

\subsection{Submodule content slide}\label{submodule-content-slide}

\begin{itemize}
\tightlist
\item
  \textbf{Aim}: Present relevant content
\item
  Highlight particularly important aspects with Quarto call-out boxes,
  for example:
\end{itemize}

\begin{tcolorbox}[enhanced jigsaw, toptitle=1mm, bottomtitle=1mm, leftrule=.75mm, bottomrule=.15mm, colframe=quarto-callout-important-color-frame, colback=white, arc=.35mm, opacitybacktitle=0.6, toprule=.15mm, coltitle=black, colbacktitle=quarto-callout-important-color!10!white, breakable, left=2mm, title=\textcolor{quarto-callout-important-color}{\faExclamation}\hspace{0.5em}{Important with Title}, opacityback=0, titlerule=0mm, rightrule=.15mm]

This is an example of a callout box to highlight particularly important
information.

\end{tcolorbox}

\begin{tcolorbox}[enhanced jigsaw, toptitle=1mm, bottomtitle=1mm, leftrule=.75mm, bottomrule=.15mm, colframe=quarto-callout-tip-color-frame, colback=white, arc=.35mm, opacitybacktitle=0.6, toprule=.15mm, coltitle=black, colbacktitle=quarto-callout-tip-color!10!white, breakable, left=2mm, title=\textcolor{quarto-callout-tip-color}{\faLightbulb}\hspace{0.5em}{Tip with Title}, opacityback=0, titlerule=0mm, rightrule=.15mm]

This is an example of a callout box to give important tips.

\end{tcolorbox}

\textbf{Speaker Notes}: Script for the slide here.

\begin{center}\rule{0.5\linewidth}{0.5pt}\end{center}

\subsection{Practical exercise 1}\label{practical-exercise-1}

\begin{itemize}
\item
  \textbf{Aim}: Design more practical exercises for students to apply
  the new skills in practice.
\item
  Depending on the topic, the exercises should be in accordance with the
  learning objective(s).
\item
  Add a description of the task, as well as a checklist as an overview
  of that your students need to be doing.
\item[$\boxtimes$]
  Step 1
\item[$\square$]
  Step 2
\item[$\square$]
  Step 3
\end{itemize}

\textbf{Speaker Notes}: Script for the slide here. \textbf{Instructor
Notes}: - \textbf{Aim}: Design more practical exercises for students to
apply the new skills in practice. - Depending on the topic, the
exercises should be in accordance with the learning objective(s). - Add
a description of the task, as well as a checklist as an overview of that
your students need to be doing.

\begin{center}\rule{0.5\linewidth}{0.5pt}\end{center}

\subsection{Pre-break survey}\label{pre-break-survey}

\begin{itemize}
\tightlist
\item
  \textbf{Aim}: This pre-break survey serves to examine students'
  current understanding of key concepts of the submodule
\item
  Use free survey software such as or other survey software (particify,
  formR) to establish the following questions (shown on separate slides)
\end{itemize}

\textbf{Speaker Notes}: Script for the slide here. \textbf{Instructor
Notes}: - \textbf{Aim}: This pre-break survey serves to examine
students' current understanding of key concepts of the submodule - Use
free survey software such as particify or formR to establish the
following questions (shown on separate slides)

\begin{center}\rule{0.5\linewidth}{0.5pt}\end{center}

\textbf{Which species is the largest type of penguin}?

\begin{enumerate}
\def\labelenumi{\alph{enumi}.}
\item
  Chinstrap Penguin
\item
  Emperor Penguin ✅
\item
  Adélie Penguin
\item
  King Penguin
\end{enumerate}

\begin{center}\rule{0.5\linewidth}{0.5pt}\end{center}

\textbf{What is the key biological feature that helps penguins swim
efficiently?}

\begin{enumerate}
\def\labelenumi{\alph{enumi}.}
\item
  Hollow bones for buoyancy
\item
  Webbed feet for paddling
\item
  Waterproof feathers and flipper-like wings ✅
\item
  Gills to breathe underwater
\end{enumerate}

\begin{center}\rule{0.5\linewidth}{0.5pt}\end{center}

\section{Break! 15 minutes}\label{break-15-minutes}

\begin{center}\rule{0.5\linewidth}{0.5pt}\end{center}

\subsection{Post-break survey
discussion}\label{post-break-survey-discussion}

\begin{itemize}
\tightlist
\item
  \textbf{Aim}: To clarify concepts and aspects that are not yet
  understood
\item
  Highlight specific answers given during the survey
\end{itemize}

\textbf{Speaker Notes}: Script for the slide here. \textbf{Instructor
Notes}: - \textbf{Aim}: To clarify concepts and aspects that are not yet
understood - Highlight specific answers given during the survey

\begin{center}\rule{0.5\linewidth}{0.5pt}\end{center}

\subsection{Practical exercise 2}\label{practical-exercise-2}

\begin{itemize}
\item
  \textbf{Aim}: Design more practical exercises for students to apply
  the new skills in practice.
\item
  Depending on the topic, the exercises should be in accordance with the
  learning objective(s).
\item
  Add a description of the task, as well as a checklist as an overview
  of that your students need to be doing.
\item[$\boxtimes$]
  Step 1
\item[$\square$]
  Step 2
\item[$\square$]
  Step 3
\end{itemize}

\textbf{Speaker Notes}: Script for the slide here. \textbf{Instructor
Notes}: - \textbf{Aim}: Design more practical exercises for students to
apply the new skills in practice. - Depending on the topic, the
exercises should be in accordance with the learning objective(s). - Add
a description of the task, as well as a checklist as an overview of that
your students need to be doing.

\begin{center}\rule{0.5\linewidth}{0.5pt}\end{center}

\subsection{Relevance and
implications}\label{relevance-and-implications}

\begin{itemize}
\tightlist
\item
  \textbf{Aim}: To work out the relevance of the topic to your students.
\item
  In an interactive setting, discuss how the new skills could be applied
  in practise with specific examples.
\item
  Examine downfalls and practical obstacles.
\end{itemize}

\textbf{Speaker Notes}: Script for the slide here. \textbf{Instructor
Notes}: - \textbf{Aim}: To work out the relevance of the topic to your
students. - In an interactive setting, discuss how the new skills could
be applied in practise with specific examples. - Examine downfalls and
practical obstacles.

\begin{center}\rule{0.5\linewidth}{0.5pt}\end{center}

\subsection{Assignment}\label{assignment}

\begin{itemize}
\tightlist
\item
  \textbf{Aim}: Explain the homework assignment and the rationale behind
  the homework.
\item
  Examine whether/how it will be assessed
\item
  Mention scoring rubrics, if applicable
\item
  Design a peer-review system for assignments to place students in role
  of reviewer and author
\end{itemize}

\textbf{Speaker Notes}: Script for the presentation here.
\textbf{Instructor Notes}: - \textbf{Aim}: Explain the homework
assignment and the rationale behind the homework. - Examine whether/how
it will be assessed - Mention scoring rubrics, if applicable - Design a
peer-review system for assignments to place students in role of reviewer
and author

\begin{center}\rule{0.5\linewidth}{0.5pt}\end{center}

\subsection{Take-home message}\label{take-home-message}

\begin{itemize}
\tightlist
\item
  \textbf{Aim}: End lesson on clear take-home message that are
  interactively compiled by students.
\end{itemize}

\begin{tcolorbox}[enhanced jigsaw, toptitle=1mm, bottomtitle=1mm, leftrule=.75mm, bottomrule=.15mm, colframe=quarto-callout-tip-color-frame, colback=white, arc=.35mm, opacitybacktitle=0.6, toprule=.15mm, coltitle=black, colbacktitle=quarto-callout-tip-color!10!white, breakable, left=2mm, title=\textcolor{quarto-callout-tip-color}{\faLightbulb}\hspace{0.5em}{Tip with Title}, opacityback=0, titlerule=0mm, rightrule=.15mm]

Add one practical tips or take-home message.

\end{tcolorbox}

\textbf{Speaker Notes}: Script for the slide here. \textbf{Instructor
Notes}: - \textbf{Aim}: End lesson on clear take-home message that are
interactively compiled by students.

\begin{center}\rule{0.5\linewidth}{0.5pt}\end{center}

\subsection{To conclude: Survey time!}\label{to-conclude-survey-time}

\begin{itemize}
\tightlist
\item
  \textbf{Aim}: This post-submodule survey serves to examine students'
  current knowledge about the sumodule's topic.
\item
  Use free survey software such as particify or formR to establish the
  following questions (shown on separate slides)
\item
  Use identical questions as in the pre-submodule survey to be able to
  directly compare
\end{itemize}

\textbf{Speaker Notes}: Script for the slide here. \textbf{Instructor
Notes}: - \textbf{Aim}: This post-submodule survey serves to examine
students' current knowledge about the sumodule's topic. - Use free
survey software such as particify or formR to establish the following
questions (shown on separate slides) - Use identical questions as in the
pre-submodule survey to be able to directly compare

\begin{center}\rule{0.5\linewidth}{0.5pt}\end{center}

\textbf{What is your level of familiarity with {[}Topic{]} (e.g., basic
concepts, terminology, or tools)?}

\begin{enumerate}
\def\labelenumi{\alph{enumi}.}
\item
  I have never heard of it before.
\item
  I have heard of it but have never worked with it.
\item
  I have basic understanding and experience with it.
\item
  I am very familiar and have worked with it extensively.
\end{enumerate}

\begin{center}\rule{0.5\linewidth}{0.5pt}\end{center}

\textbf{Which of the following concepts or skills do you feel most
confident about in relation to {[}Topic{]}? (Select all that apply)}

\begin{enumerate}
\def\labelenumi{\alph{enumi}.}
\item
  Concept 1
\item
  Concept 2
\item
  Concept 3
\item
  Concept 4
\item
  I am not sure about any of these concepts.
\end{enumerate}

\begin{center}\rule{0.5\linewidth}{0.5pt}\end{center}

\textbf{On a scale of 1 to 5, how comfortable are you with using
{[}specific tool/technology{]} related to {[}Topic{]}? (1 = Not
comfortable at all, 5 = Very comfortable)}

\begin{enumerate}
\def\labelenumi{\alph{enumi}.}
\item
  1
\item
  2
\item
  3
\item
  4
\item
  5
\end{enumerate}

\begin{center}\rule{0.5\linewidth}{0.5pt}\end{center}

\subsection{Discussion of survey
results}\label{discussion-of-survey-results-1}

\begin{itemize}
\tightlist
\item
  \textbf{Aim}: Briefly examine the answers given to each question
  interactively with the group.
\item
  Compare and highlight specific differences in answers between pre- and
  post-survey answers
\end{itemize}

\textbf{Speaker Notes}: Script for the slide here. \textbf{Instructer
Notes}: - \textbf{Aim}: Briefly examine the answers given to each
question interactively with the group. - Compare and highlight specific
differences in answers between pre- and post-survey answers

\begin{center}\rule{0.5\linewidth}{0.5pt}\end{center}

\subsection{References}\label{references}

\begin{itemize}
\tightlist
\item
  Provide literature you refer to throughout this lesson.
\end{itemize}

\textbf{Speaker Notes}: Script for the slide here. \textbf{Instructor
Notes}: Highlight particularly relevant reading for your students in
bold.

\begin{center}\rule{0.5\linewidth}{0.5pt}\end{center}

\section{\texorpdfstring{Thanks! }{Thanks!  }}\label{thanks}

See you next class :)

\begin{center}\rule{0.5\linewidth}{0.5pt}\end{center}

\subsection{Pedagogical add-on tools for
instructors}\label{pedagogical-add-on-tools-for-instructors}

\begin{itemize}
\tightlist
\item
  This section is dedicated to ideas on how to incorporate pedagogical
  tools into teaching for this specific submodule topic. This could
  mean:

  \begin{itemize}
  \tightlist
  \item
    Information about the scientific evidence on the theory of the
    pedagogical add-on tool and the evidence for its efficacy.
  \item
    Discussion/reflection on how tools can be incorporated into the
    teaching for this particular content.
  \item
    Extra exercises for faster students.
  \end{itemize}
\end{itemize}

\begin{center}\rule{0.5\linewidth}{0.5pt}\end{center}

\subsection{Additional literature for
instructors}\label{additional-literature-for-instructors}

\begin{itemize}
\tightlist
\item
  References for content
\item
  References for pedagogical add-on tools\\
\item
  Other resources (videos etc.)
\end{itemize}

\begin{center}\rule{0.5\linewidth}{0.5pt}\end{center}

\section{Formatting elements for
instructors}\label{formatting-elements-for-instructors}

\begin{itemize}
\tightlist
\item
  \textbf{Aim}: This section contains templates for different formatting
  elements, which can be modified and adapted for the instructor's
  individual purposes.
\end{itemize}

\begin{center}\rule{0.5\linewidth}{0.5pt}\end{center}

\subsection{Text with example links}\label{text-with-example-links}

\begin{itemize}
\tightlist
\item
  \href{https://quarto.org/docs/}{Quarto Documentation}
\item
  \href{https://revealjs.com/}{Reveal.js Documentation}
\item
  \href{https://www.markdownguide.org/}{Markdown Guide}
\item
  \href{https://github.com/}{GitHub}
\end{itemize}

\begin{center}\rule{0.5\linewidth}{0.5pt}\end{center}

\subsection{Basic text formatting}\label{basic-text-formatting}

\begin{itemize}
\item
  \textbf{Bold:} \texttt{**bold**} → \textbf{bold}
\item
  \emph{Italic:} \texttt{*italic*} → \emph{italic}
\item
  \st{Strikethrough:}
  \texttt{\textasciitilde{}\textasciitilde{}text\textasciitilde{}\textasciitilde{}}
  → \st{text}
\item
  \texttt{Inline\ code:} \texttt{\textasciigrave{}code\textasciigrave{}}
  → \texttt{code}
\item
  \begin{quote}
  Blockquote: \texttt{\textgreater{}\ Quote} →\\
  ``This is a quote''
  \end{quote}
\end{itemize}

\begin{center}\rule{0.5\linewidth}{0.5pt}\end{center}

\subsection{Figure with caption}\label{figure-with-caption}

\begin{itemize}
\tightlist
\item
  Centered image and caption below in italics
\end{itemize}

This is a Penguin.

\begin{center}\rule{0.5\linewidth}{0.5pt}\end{center}

\subsection{Figure with bullet points}\label{figure-with-bullet-points}

\begin{itemize}
\tightlist
\item
  First bullet point
\item
  Second bullet point
\item
  Third bullet point
\end{itemize}

\begin{center}\rule{0.5\linewidth}{0.5pt}\end{center}

\subsection{Side-by-side figures}\label{side-by-side-figures}

\begin{center}\rule{0.5\linewidth}{0.5pt}\end{center}

\subsection{Stacked figures with text}\label{stacked-figures-with-text}

\begin{itemize}
\tightlist
\item
  First bullet point
\item
  Second bullet point
\item
  Third bullet point
\end{itemize}

\begin{center}\rule{0.5\linewidth}{0.5pt}\end{center}

\subsection{Two-column text slide}\label{two-column-text-slide}

\textbf{Column 1}

Lorem ipsum dolor sit amet, consectetur adipiscing elit.\\
Vivamus lacinia odio vitae vestibulum vestibulum.\\
Cras venenatis euismod malesuada.

\textbf{Column 2}

Sed do eiusmod tempor incididunt ut labore et dolore magna aliqua.\\
Ut enim ad minim veniam, quis nostrud exercitation ullamco laboris.

\begin{center}\rule{0.5\linewidth}{0.5pt}\end{center}

\subsection{Three-column text slide}\label{three-column-text-slide}

\textbf{Column 1}

Lorem ipsum dolor sit amet, consectetur adipiscing elit.\\
Vivamus lacinia odio vitae vestibulum vestibulum.

\textbf{Column 2}

Sed do eiusmod tempor incididunt ut labore et dolore magna aliqua.\\
Ut enim ad minim veniam, quis nostrud exercitation ullamco laboris.

\textbf{Column 3}

Duis aute irure dolor in reprehenderit in voluptate velit esse cillum
dolore eu fugiat nulla pariatur.

\begin{center}\rule{0.5\linewidth}{0.5pt}\end{center}

\subsection{Simple table}\label{simple-table}

\begin{longtable}[]{@{}lll@{}}
\toprule\noalign{}
Column 1 & Column 2 & Column 3 \\
\midrule\noalign{}
\endhead
\bottomrule\noalign{}
\endlastfoot
Row 1 Cell & Row 1 Cell & Row 1 Cell \\
Row 2 Cell & Row 2 Cell & Row 2 Cell \\
Row 3 Cell & Row 3 Cell & Row 3 Cell \\
Row 4 Cell & Row 4 Cell & Row 4 Cell \\
\end{longtable}

\begin{center}\rule{0.5\linewidth}{0.5pt}\end{center}

\subsection{Complex table}\label{complex-table}

\begin{longtable}[]{@{}lll@{}}
\toprule\noalign{}
Column 1 & Column 2 & Column 3 \\
\midrule\noalign{}
\endhead
\bottomrule\noalign{}
\endlastfoot
Row 1 Cell & Row 1 Cell & Row 1 Cell \\
Row 2 Cell & Row 2 Cell & Row 2 Cell \\
Row 3 Cell & Row 3 Cell & Row 3 Cell \\
Row 4 Cell & Row 4 Cell & Row 4 Cell \\
\end{longtable}

\begin{center}\rule{0.5\linewidth}{0.5pt}\end{center}

\subsection{Task list}\label{task-list}

\begin{itemize}
\tightlist
\item[$\boxtimes$]
  Done
\item[$\square$]
  To do
\end{itemize}

\begin{center}\rule{0.5\linewidth}{0.5pt}\end{center}

\subsection{Embedding videos}\label{embedding-videos}

\begin{center}\rule{0.5\linewidth}{0.5pt}\end{center}

\subsection{Code blocks}\label{code-blocks}

\begin{Shaded}
\begin{Highlighting}[]
\CommentTok{\# A basic R code chunk}
\NormalTok{x }\OtherTok{\textless{}{-}} \DecValTok{1}\SpecialCharTok{:}\DecValTok{10}
\FunctionTok{mean}\NormalTok{(x)}
\end{Highlighting}
\end{Shaded}

\begin{verbatim}
[1] 5.5
\end{verbatim}

\begin{Shaded}
\begin{Highlighting}[]
\CommentTok{\# A simple plot}
\FunctionTok{hist}\NormalTok{(}\FunctionTok{rnorm}\NormalTok{(}\DecValTok{100}\NormalTok{), }\AttributeTok{main =} \StringTok{"Histogram of Random Normals"}\NormalTok{)}
\end{Highlighting}
\end{Shaded}

\pandocbounded{\includegraphics[keepaspectratio]{OS-template-slides_files/figure-pdf/unnamed-chunk-1-1.pdf}}

\begin{center}\rule{0.5\linewidth}{0.5pt}\end{center}

\subsection{License}\label{license}

\begin{itemize}
\item
  This slide should contain information about the license details of
  this current set of slides.
\item
  The default for the created materials is
  \href{https://creativecommons.org/licenses/by/4.0/}{CC-BY-SA 4.0}

  \begin{itemize}
  \tightlist
  \item
    = Creative Commons license that allows others to \textbf{share,
    adapt, and build upon} the original work
  \item
    \textbf{only} if they attribute the creator and also share their new
    work under the \textbf{same terms}
  \item
    allows for both \textbf{commercial and non-commercial} use of the
    licensed material
  \end{itemize}
\item
  Components of attributions:

  \begin{itemize}
  \tightlist
  \item
    \texttt{Title}
  \item
    \texttt{Author}
  \item
    \texttt{Source}
  \item
    \texttt{License}
  \end{itemize}
\end{itemize}

\begin{center}\rule{0.5\linewidth}{0.5pt}\end{center}

\subsection{CREDiT Contribution
Statement}\label{credit-contribution-statement}

Possible roles using the CRediT contribition system or the Zenodo
Contribution System:

\textbf{Name main content creator}: Conceptualization, Software, Writing
- Original Draft, Visualization. \textbf{Sara Lil Middleton}: Writing -
Review \& Editing, Supervision. \textbf{Sarah von Grebmer zu
Wolfsthurn}: Conceptualisation, Writing - Review \& Editing,
Supervision, Project Administration, Validation.

\begin{center}\rule{0.5\linewidth}{0.5pt}\end{center}




\end{document}
